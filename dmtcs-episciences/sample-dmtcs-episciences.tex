\documentclass[
%%% one of
submission
%final
%proceedings
%%% if you compile a final version for the old OJS platform
% , ojs
%%% if all authors have the same affiliation
% , nomarks
]{dmtcs-episciences}

% DON'T LOAD ANY STYLES THAT CHANGE THE PAGE LAYOUT
% AND DON'T CHANGE THE PAGE LAYOUT BY HAND, EITHER.


\usepackage[utf8]{inputenc}
\usepackage{subfigure}

% graphicx is now loaded automatically no need to put this in here anymore.
%
%\usepackage{graphicx}

% We strongly recommend to use natbib. Your colleagues deserve to be
% named in your text. PLEASE, ADAPT YOUR TEXT ACCORDINGLY, such that
% citations are grammatically correct.
\usepackage[round]{natbib}

\author{Jens Gustedt\affiliationmark{1,2}\thanks{I am fully supported.}
  \and Somebody Who\affiliationmark{3}\thanks{And he is, too!}
  \and Some Dummy\affiliationmark{3}}
\title[Formatting an article for DMTCS]{How to format an article
  for DMTCS\\
  with the journal's own \LaTeX-style}
% put your affiliation here, not your full address.
% If you like to give away your email or other parts of your address,
% THIS IS NOT THE RIGHT PLACE, your address will change, this paper
% will not.
% Just watch that your personal data that you want to communicate on
% the episcience server is always up to date.
\affiliation{
  % one line per affiliation, no postal codes, grant numbers or similar
  INRIA, France\\
  ICube, university of Strasbourg, France\\
  Alma Mater, campus universalis, terra incognita}
\keywords{some, well classifying, words}
% don't try to cheat here, we will check the dates!
\received{1998-10-14}
\revised{2002-07-19, 2014-02-05, 2015-09-09}
\accepted{2015-09-09}
\begin{document}
\publicationdetails{VOL}{2015}{ISS}{NUM}{SUBM}
\maketitle
\begin{abstract}
  DMTCS is an open access scientific journal that is build on top of
  public archives. It is implemented by the \emph{episcience}
  platform, see Berthaud et al., 2014, for an overview of the
  strategy. One implication of that strategy is that authors have to
  provide good quality \LaTeX{} for their articles themselves. This
  paper here is supposed to provide you with the necessary tools to do
  so. \emph{Please read it carefully.}
\end{abstract}

DMTCS is an open access scientific is implemented by the
\emph{episcience} platform, see \cite{berthaud:hal-01002815} for an
overview of the strategy. It combines high scientific and editorial
quality with an open access policy. It is priceless, neither authors
nor readers pay money for the access. Access is granted by giving
episcience an irrevocable license to publish the articles, the
copyright remains with the authors. The platform itself is run by
French government services that do their best to warrant continuous
access and a high quality of service.

This document describes the use of the \texttt{dmtcs-episcience.cls}
document class. It should be used
\begin{center}
  \emph{\textbf{for all DMTCS publications}}.
\end{center}

If you are still preparing a document for our previous \texttt{OJS}
platform, please add \texttt{ojs} to the classes options, see
Section~\ref{sec:options}.

\section{Introduction}
\label{sec:in}
These pages here show you how to prepare your final version of a
manuscript before publication, once you have received the OK from the
editorial board. Nevertheless, we encourage you to prepare it already
according to these hints here during the submission phase. In fact,
what we ask from you isn't really so special. It should even be useful
if your paper doesn't make it into DMTCS, and you later submit it to
some other scientific journal.

The \LaTeX-style for DMTCS is derived straight forward from the
usual \texttt{article.sty}. Its main purpose is to ensure a common
layout policy of all articles in DMTCS and to provide editors,
referees and readers with the necessary information. If you think you
need an introduction to \LaTeX{} or search for pointers to other
literature on that, you should consider
reference~\cite{oetiker99:_not_so_short_introd_latex} given at the
end.

Before you read on to know how to use our style file please ensure
that the manuscript fits well into DMTCS.
\begin{enumerate}
\item DMTCS covers \emph{Discrete Mathematics} and \emph{Theoretical 
    Computer Science} as domain of interest. 
\item DMTCS is a \emph{scientific journal}. This requires that your
  work is

  \begin{description}
  \item[original] The main results of your paper must not have
    appeared elsewhere in a journal, neither by yourself nor by
    somebody else.

    There is no excuse for plagiarism, not even self-plagiarism. We
    have a good record in tracking such things down, be warned.
  \item[important] Your results must be of importance to a wider
    public and should be of interest for more readers than just the
    referees. 
  \item[self contained] Other than for conference proceedings, we
    don't have any restrictions on the number of pages for an
    article. So there is no excuse to suppress proofs or not to give
    verbose examples. On the other hand, self containing short notes
    are highly welcome.
  \item[complete] Your work must relate to the current state of the
    art of the domain in question. In particular, foreign and own
    results external to the manuscript must be correctly credited and
    complete reference to such related work must be given. 
  \item[legible] The journal's language is \emph{English}, all
    conventions for scientific work in that language apply. 

  \item[correct] Your work must be mathematical correct \emph{and}
    its quality of writing must be such that the referees will agree
    upon this fact.

    Your writing must be grammatically correct. Be ensured, that
    especially authors that are non-native speakers of English will
    receive all possible help to correct flaws. But also have in mind,
    that incorrect grammar might be the cause of severe
    misunderstandings and finally result in a rejection of the
    paper.  
  \end{description}
\item DMTCS is 
  \begin{description}
  \item[no] circular letter, and
  \item[no] preprint server.
  \end{description}
  If you are looking for that, please consider the wide
  possibilities that the web offers nowadays.
\end{enumerate}

% You may scarsely use \clearpage to advance to a new page if this
% improves the readability of the document structure
\clearpage
\section{Providing Information for the first Page}
\label{sec:first}

First of all, for a correct submission we need some basic information.
Consider this file here itself as an example how this should be done.
We need the following type of information:

\begin{itemize}
\item The name(s) of the author(s), provided by the \verb!\author!
  command.

  This is about the same as for standard \LaTeX. Please refer to
  your \LaTeX\ book to see how this is usually done, or look at the
  examples given here in this file. 
  
  \begin{center}
    \emph{Don't put homebrewn \LaTeX-commands or special characters here.}
  \end{center}

  They will disturb the extraction of the meta-data. Please write out
  everything you may, in particular encode accented characters with
  \LaTeX-commands, e.g \verb!{\'e}! for an `{\'e}'.

  \begin{description}
  \item[\emph{and}] If there is more than one author, the names are
    separated by \verb!\and! commands. If all fits well on one line,
    the author names should automatically be spread out equally over
    that line. If there are too many, place a line terminating
    \verb!\\!  after the last name of an author line. The next line
    then should start with the next \verb!\and!.
  \item[\emph{mark(s)}] If authors have different affiliations, put an
    appropriate \verb!\affiliationmark{i}!  \emph{directly} after the
    last name of an author, where $i$ corresponds to affiliation
    number $i$. See below, on how to include several affiliations into
    the affiliation command.

    If all authors share one or all affiliations, there is no need to
    have the \verb!\affiliationmark{i}! since it is redundant. In that
    case, please add \verb!nomarks! to the options of the document
    class, see below.
  \item[thanks] If individual authors have to express their gratitude
    towards a particular grant institution, a \verb!\thanks! command
    that immediately follows the name is the right place.
    \begin{itemize}
    \item There should only be one \verb!\thanks! per author.
    \item It should not contain address information such as phone
      numbers or email addresses.
    \item If the text concerns all of the paper, then the appropriate
      place for the \verb!\thanks! is the title, not the author name.
    \end{itemize}

  \end{description}
\item The title of the manuscript, provided by the \verb!\title!
  command.

  The title command may be given in two different forms. The first is 
\begin{verbatim}
\title{Your title goes here}
\end{verbatim}
  If done like that, the title that you give is used as running head
  for the odd numbered pages as well. If your title is too long such
  that it doesn't fit into the running head you should use the
  alternative form
\begin{verbatim}
\title[Formatting a submission for DMTCS]{How to format a submission 
for DMTCS with the journal's own \LaTeX-style}
\end{verbatim}
  Here the string that inside the \verb![ ]! is used in the
  running head.

  The same rule as for the author command applies here, don't use
  avoidable specialties in here.

  Use \verb!\thanks! to express gratitude to people or organizations
  that help the paper as a whole, not only just one particular
  author.
\item The affiliation(s) of the author(s), provided by the \verb!\affiliation!
  command.

  If authors have different affiliations or several of them, put each
  such affiliation on a line of its own, separate the lines by a
  \verb!\\!  command. For an example, look at the \LaTeX-source of
  this file here. As you can see I am listed for two different
  institutions. These appear on separate lines.

  The affiliations will appear as numbered lines in the order they are
  given. These numbers are the ones that you should use for the
  \verb!\affiliationmark! command for the authors that we have seen above.

  The text here should be the way how your institutions is usually
  cited and identified by others, this is \emph{not} the postal
  address, nor should it contain cryptic grant numbers, lab counters
  or anything of that kind.

  If you have to refer to such things as grant or lab numbers, that is
  things that are usually not interesting for your readers by for your
  controlling or evaluation institutions, put them in a \verb!\thanks!
  command.

  This also isn't a good place to give away your address, be it
  postal, email or phone number. Addresses change, papers
  shouldn't. Your episciences account gives you the opportunity to
  have all your address data up-to-date, always.
\item Some keywords that classify your work, provided by the
  \verb!\keywords!  command. This is mandatory. Be careful on the
  choice of these keywords, you are the author, you should know best
  what is adequate such that your article can be easily and correctly
  identified by search engines and alike. Give it in the form
\begin{verbatim}
\keywords{first item, second, third}
\end{verbatim}
  So each ``\emph{key word}'' might consist of several words in the
  usual sense. To separate several key words use commas.

  These keywords must be the same as the ones that are given when you
  fill out the http-form for submission.

  The same rule as for the author command applies here, don't use
  avoidable specialties in here.
\item Most of the editorial details for the paper come with the
  \verb!\publicationdetails! command. It has five arguments, for the
  volume, the year, the issue, the number of the paper and the
  original submission ID. You should have received information what to
  put here when you were notified about the acceptance of your
  paper. If you are using the DMTCS \texttt{.cls} file for a
  submission, just update the date and otherwise leave it as it is.

  In addition to that, some dates of the reviewing procedure should
  also be given with \verb!\received!, \verb!\revised! and
  \verb!\accepted!, respectively. Dates must be of the form
  \texttt{yyyy-mm-dd}. If there have been several revisions, separate
  the dates with commas.
\item
  An abstract of you manuscript, provided by the \verb!abstract!
  environment. This should be no longer than a paragraph and concisely
  reflect the main contributions of your work.

  This abstract should be some brief description of
  your work and how it advances our science.
  \begin{description}
  \item[Don't] put complicated mathematical expressions here.
  \item[Don't] use commands you defined yourself here.
  \item[Don't] put \verb!\cite! commands here. If you feel that a
    reference is important for the overall estimation of your work,
    spell that reference out by using something like
    ``\emph{This result has been conjectured by Erd\H{o}s et al. (1931)}''.
  \end{description}
  The basic rule is that this abstract must be ``understandable by its
  own'' and this both for its contents \emph{and} for its
  TeXnicalities.

  Many readers (such as editors) base their selection whether to look
  at a paper more closely on that abstract. In particular there are
  high chances that the decision which referees are assigned to your
  manuscript is mainly based on that abstract. \emph{You have been
    warned.}
\end{itemize}


\section{Hints for the manuscript itself}
\label{sec:hints}
We suppose in the following that you write a paper since you want to
\emph{publish} it, \textit{i.e.}, make it publicly available, and
that you want it to be read and understood. Therefore it is imperative
that you stay inside the established conventions for mathematical or
TCS texts. People are used to these conventions.  They help them to
easily and quickly access the real contents of your text and to not to
be diverted by its appearance.

\subsection{Numbering commands}
\label{sec:numbering}

Please use the standard conventions for all commands and environments
that provide a numbering such as theoremlike environments or sections.
In particular usual counting starts at $1$ and not at $0$. An
introduction is an integral part of a paper and should be counted as
one ($=1$!).

\begin{center}
  \emph{Never ever number items, paragraphs, equations, cases theorems
    or whatsoever manually.}
\end{center}

This is the age of computers, use them before they use you. \LaTeX{}
automatically produces numbers to put hyperlinks into your text, such
that reading a paper in DMTCS becomes a real `electronic' experience.

If you have a case analysis that goes over several pages, the
extensions of the \verb!enumerate! package can be interesting. It is
as simple as the following\\[3ex]

\hfil
\begin{minipage}{0.45\linewidth}
\begin{verbatim}
% in the preable
\usepackage{enumerate}
...
...
% in the document itself
\begin{enumerate}[{Case }1:]
\item when all begins ...
\item when it all ends
\end{enumerate}
\end{verbatim}
\end{minipage}%
\hfil
\begin{minipage}{0.45\linewidth}
\vspace*{12ex}
\begin{enumerate}[{Case }1:]
\item when all begins ...
\item when it all ends
\end{enumerate}
\end{minipage}\\[3ex]




\subsection{Markup commands}
\label{sec:markup}

Don't use markup of text according to some layout or style, but to
stress semantic differences. The correct way to \emph{emphasize} a
certain part of your text is \verb!\emph{emphasize}!. Don't use the
\verb!\text..! family for that purpose, in particular don't use
\verb!\textit! or \verb!\it!. \emph{They have \emph{different}
  meanings} and rendering. \textit{E.g.}, observe the word
\emph{different} in the previous phrase: this is rendered in an
upright font since this is an \verb!\emph! inside another
\verb!\emph!. On the other hand \verb!\texit! for the abbreviations
``\textit{e.g.}'' and ``\textit{i.e.}'' is appropriate. This is
because they are Latin (for \textit{exempli gratia} and \textit{id
  est}), and all foreign text inside English text has to be put in
italics.

If ever you use commands that change the font use the modern form
\verb!\text..! for them, such as \verb!\textbf{text}!,
\verb!\textit{text}! or \verb!\textsc{text}!. These commands know
better what has to be done when they switch back to normal than the
ancient commands \verb!{\bf text}!, \verb!{\it text\/}! or
\verb!{\sc text}!.

\subsection{Headings}
\label{sec:headings}
Use the standard heading and structuring commands \verb!\section!,
\verb!\subsection! \textit{etc.} to structure your document. For
theorems use the corresponding environments that you may define by
means of the \verb!\newtheorem! command.

\subsection{Proper Names}
\label{sec:names}
Names of theorems and alike are considered to be proper names. In
English (!) proper names are capitalized. So please write
something like ``\emph{In Section~\ref{sec:first} we have seen...}''
and ``\emph{By the Main~Theorem we know...}''. But distinguish
properly from the use of the word ``theorem'' as ordinary noun as it
is for example in ``\emph{In the following theorem we prove ...}''.

Please also be careful in the writing of personal names. Customs in
different countries are different! Be sure to use a standard
transcription of names that use a different alphabet than English, and
also be sure to use the full capabilities of \LaTeX{} for accentuated
character sets that are based on the Latin alphabet. Be sure to catch
the correct concept of ``last name'' in that language.


\subsection{Use a Spell Checker}
\label{sec:check}

It is considered as being very impolite to leave obvious spelling
errors in the manuscript before sending it out. Computers are made for 
these, \emph{use them}.

You might either use the North American variant for spelling or the
British one, but please don't mix them in one paper. The same holds
for different possible spellings for the same word as for example
``\emph{acknowledg(e)ment}'' or ``\emph{formulae}'' versus
``\emph{formulas}''. Be coherent.

\subsection{Mathematics}
\label{sec:math}
Running text must always constitute correct English phrases.  An
English phrase needs a verb and an `$=$'-sign can not be a replacement
for it.
  
All complicated mathematical formulas should be given on separate
lines and should not be spread out into the running text.  Never use
the \verb!$! form of the math environment for these. Human or automatic
taggers have a hard time to recognize which is an opening or a closing
\verb!$!. Use
\begin{verbatim}
\begin{math}...\end{math} 
\end{verbatim}
for all formulas that spread over several words and
\begin{verbatim}
\begin{displaymath}...\end{displaymath}
\end{verbatim}
(or \texttt{equation} \textit{etc.}) that should be rendered on a line
of their own. Using the old fashioned double-dollar environment
\begin{verbatim}
$$ some complicated formula $$ 
\end{verbatim}
is frowned upon.

You should use \LaTeX{} environments that provide a numbering for all
formulas that are rendered on line of the their own. Use environments
such as \texttt{equation} or \texttt{eqnarray}. Such numbers ease the
referee process very much, and after eventual publication easily allow
readers to refer to in their own work.
  
  
The quantifiers ``$\exists$'' and ``$\forall$'' don't stand as abbreviations of
the partial phrases ``\emph{there is}'' and ``\emph{for all}''. They
are reserved for logical formulas as \emph{such}, that is for work
that talks itself of logical formulas as a subject.
  
  
The equal sign ``$=$'' has different meanings in parts of the two
communities that DMTCS addresses.
\begin{enumerate}
\item It might stand for mathematical identity that is discovered
  \emph{a posteriori}. As an example take the following phrase:
  \begin{center}
    \emph{An easy computation shows that $4!=24$}.
  \end{center}
\item It might stand for a \emph{definition}, as in
  \begin{center}
    \emph{For convenience, set $0!=1$.}
  \end{center}
\end{enumerate}
For the later use of ``$=$'' Computer Scientist often tend to use
``$:=$''. Referees should be tolerant to these different customs.


\subsection{Proofs}
\label{sec:proofs}

Put all proofs that you provide in your article in a \texttt{proof}
environment, such as this
\begin{verbatim}
\begin{proof}
  Use this for proofs ...
\end{proof}
\end{verbatim}

\begin{proof}
  Use this for proofs. This helps the reader to understand
  the general structure of your paper easily. In particular, we easily
  see where the proof ends.
\end{proof}

\begin{proof}[of Main Theorem]
  Use this form, if the proof doesn't follow directly after the
  theorem so you have to refer it by name or number.
\end{proof}


\subsection{Cross-references and citations}
\label{sec:cross}
Use cross-references throughout your whole paper. Use \verb!\label! and
\verb!\ref! for that and don't do the work of the computer by
yourself. Not only that it is easier (\emph{believe me!}) but also
it helps to insert hyperlinks across the final document in the pdf
version, see Section~\ref{sec:pdf}. 

The same holds for citations.
\begin{center}
  \emph{Never ever number citations by hand.}
\end{center}
This only can go wrong and it will. Use \LaTeX{}' \verb!\cite!
command. Again, in the pdf version this will have the advantage of a
hyperlink that lets you jump directly to the bibliography item.

Use \texttt{bibtex} to produce your bibliography. With a little bit of
initial overhead it lets you easily maintain your references. This
pays off when you will write more than one article in your life...
Have a look into \cite{oetiker99:_not_so_short_introd_latex} and to
the
\href{http://www.dmtcs.org/abstracts/bib/DMTCS.bib}{\texttt{.bib}-file}
of DMTCS to see how this works.

I personally prefer the so-called \emph{natural} citation style as it
is used herein (via \texttt{natbib}). It has the advantage that the
author names of the work that is cited appear properly. Papers are to
the merits of people. In addition, such a citation by
name has the advantage of being easily recognizable without looking in
the bibliography. 


\section{PDF Files}
\label{sec:pdf}
\texttt{PDF} has become the de-facto standard for scientific documents
in our domains. To produce that format with our style supports
\texttt{PDF} you would typically use \texttt{pdflatex}
as the formatting command. If you are viewing this document in its pdf
form you may see some of the advantages this has: in particular pdf
documents produced in that way have included \emph{hyperlinks}. If you
want to know more about these features please refer to
\href{http://xxx.lanl.gov/hypertex/}{\texttt{http://xxx.lanl.gov/hypertex/}}.

The switching between pdf or the traditional dvi format should be done
automatically on whether you process your file with \texttt{pdflatex}
or \texttt{latex}. If this doesn't work for you you could try to add
the option \texttt{pdflatex} to the \verb!\documentclass! right at the
start of the document.

Only very old \LaTeX{} installations don't support the package
\texttt{hyperref} these days. If you have such an old installation you
should seriously think of switching to a newer one. It will pay off.

\begin{figure}[htbp]
  \begin{center}
    %% Don't use epsfile!
    \includegraphics{dmtcs}
    \caption{The logo of \protect\href{http://www.dmtcs.org/}{DMTCS}.}
    \label{fig:logo}
  \end{center}
\end{figure}
\section{Graphics}
\label{sec:graphics}

Please use the (standard) packages \texttt{graphics} or
\texttt{graphicx} to include graphical data and not
\texttt{epsf} or similar. Something like the PostScript picture in the
title of this document can be produced as simple as this
\verb!\includegraphics[width=0.13\textwidth]{dmtcs}!. Note that in
this command the width is given in relation to the width of the text
and not in an absolute measure and that the file name is given without
extension. 
\begin{center}
  \emph{Don't include the extension of the graphic file in the
    command!}
\end{center}
Please, leave the choice of the desired format to the
\texttt{graphicx} package.


For a realistic graphic of your paper you should chose a
\texttt{figure} environment as is done with the following for 
Figure~\ref{fig:logo} 
\begin{verbatim}
\begin{figure}%[htbp]
  \begin{center}
    %% Don't use epsfile!
    \includegraphics{dmtcs}
    \caption{The logo of DMTCS.}
    \label{fig:logo}
  \end{center}
\end{figure}
\end{verbatim}

If you have several (small) figures that are related to each other you may
place them inside one figure environment. As you may see in
Figure~\ref{fig:logo2} how this may look like and consider the source
of this description here on how to refer to each part (that is
Figure~\ref{logo1cm} and \ref{logo3cm}) individually.
\begin{figure}[htbp]
  \begin{center}
    \subfigure[height of 1 cm\label{logo1cm}]{\includegraphics[height=1cm]{dmtcs}}
    \hfil
    \subfigure[width of 3 cm\label{logo3cm}]{\includegraphics[width=3cm]{dmtcs}}
    \caption{The scaled logo of \protect\href{http://www.dmtcs.org/}{DMTCS}.}
    \label{fig:logo2}
  \end{center}
\end{figure}

\section{Summary of Options}
\label{sec:options}
\begin{tabular}{|l|p{8cm}|}
\hline
option & description\\
\hline
submission & whether or not this is considered as a submission or\\
final & being the final document for journal papers\\
proceedings & or proceedings volumes\\
ojs & if you are providing a paper for our old OJS platform\\
\hline
nomarks & if all authors have the same affiliation(s)\\
\hline
pdftex & force production of pdf, if \texttt{pdflatex} doesn't work\\
nohyperref & switch reference to \texttt{hyperref} off\\
notimes & switch selection of the \texttt{times} package off\\
\hline
\end{tabular}


\section{Summary of Relevant Commands}
\label{sec:commands}

\begin{tabular}{|l|p{10cm}|}
\hline
command & description\\
\hline
\verb!\affiliation! & the affiliation of the authors\\
\verb!\affiliationmark! & to number different affiliations for different authors\\
\hline
\verb!\publicationdetails{V}{Y}{I}{N}{ID}! & In the final
version: the information to identify your paper\\
\verb!\received{YYYY-MM-DD}! & In the final version: the date of submission to DMTCS.\\
\verb!\revised{YYYY-MM-DD}!  & In the final version: the date(s) of revisions.\\
\verb!\accepted{YYYY-MM-DD}! & In the final version: the date of acceptance by the
editorial board.\\
\hline
\verb!\keywords! & a comma separated list of keywords\\
\hline
\verb!\qed! & produces \qed. See \texttt{proof}-environment below.\\
\hline
\verb!\acknowledgements! & put an acknowledgment section at the end\\
\hline
& Use the following commands only for the indicated
  purpose. If, \textit{e.g}, you wand to use a $\mathbb{P}$ that does
  not denote the set of prime numbers use \verb!\mathbb{P}!\\
\hline
\verb!\naturals! & the positive integers or `naturals', \naturals\\
\verb!\integers! & the ring of the integers, \integers\\
\verb!\rationals! & the field of the rational numbers, \rationals\\
\verb!\reals! & the field of the real numbers, \reals\\
\verb!\complexes! & the field of the complex numbers, \complexes\\
\verb!\primes! & the set of prime numbers, \primes\\
\hline
\end{tabular}

\section{Summary of Relevant Environments}
\label{sec:environments}

\begin{tabular}{lp{8cm}}
  \texttt{abstract} & The abstract of your paper, see Section~\ref{sec:first}.\\
  \texttt{proof} & Use this for proofs, see Section~\ref{sec:proofs}
\end{tabular}

\acknowledgements
\label{sec:ack}
At the end of the manuscript, right before the bibliography you might
want to place an acknowledgment. This can be easily done by using the 
command \verb!\acknowledgements! as you can see here.

\nocite{*}
\bibliographystyle{abbrvnat}
% use the following instead if you encounter problems 
%\bibliographystyle{alpha}
\bibliography{sample-dmtcs}
\label{sec:biblio}

\end{document}
