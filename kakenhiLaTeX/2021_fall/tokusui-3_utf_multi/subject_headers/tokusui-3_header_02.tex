\documentclass[8pt]{extarticle}
\usepackage{newtxtext,newtxmath} % Times New Roman
\usepackage[top=0pt, bottom=0pt]{geometry}
\setlength{\oddsidemargin}{-8pt}
\setlength{\evensidemargin}{-8pt}
\usepackage{tabularx}
\usepackage{mathptmx}
\usepackage{array}
%https://tex.stackexchange.com/questions/41758/how-can-i-reproduce-this-table-with-thick-lines
\makeatletter
\newcommand{\thickhline}{%
	\noalign {\ifnum 0=`}\fi \hrule height 1pt
	\futurelet \reserved@a \@xhline
}
\newcolumntype{"}{@{\vrule width 1pt}}
\makeatother
\usepackage{graphicx}

\begin{document}	
\noindent\textbf{\fontsize{12}{12}\selectfont Items to be Entered When New Application is Made in the Fiscal Year Previous to the Final Year of the Research Period of an On-Going KAKENHI Project}\\
\indent \resizebox{1.05\textwidth}{!}{(For an application that comes under this category, this column is a mandatory entry. (\textit{cf.} Application Procedures for Grants-in-Aid for Scientific Research))}\\
\begin{tabularx}{1.1\linewidth}{"|X|"}
	\thickhline
	The applicant should give within 1 page: (1) the relevant information on the on-going project (for which FY2022 is the final year of the research period) including the original plan at the time of application/adoption and the research accomplishment such as new knowledge acquired, and (2) the reason why he/she is submitting this new proposal for FY2022 on top of the on-going project (in terms of the development of the on-going research, necessity of new research budget, etc.). If not applicable, leave this page blank. (Do not eliminate the page.) 
	\\
	\thickhline
\end{tabularx}
\end{document}

