% form04_tokusui_s-3_header.tex
%=================================================================

% ===== Global definitions for the Kakenhi form ======================
% 基本情報
\newcommand{\様式}{様式S−1(3)}
\newcommand{\研究種目}{特別推進研究}
\newcommand{\研究種目後半}{ 新規(日本語)}
\ifthenelse{\isundefined{\研究種別}}{%
	\newcommand{\研究種別}{}
}{}%
\newcommand{\研究種目header}{\研究種目}

\newcommand{\KLMainFile}{jpa\_tokusui-3.tex}
\newcommand{\KLYoshiki}{tokusui-3_header}

% ===== Headers =====================================

\fancypagestyle{tokusui-2-10-header}{%
	\fancyhf{}
	\fancyhead[L]{\hspace{-37pt}\headerfont{様式S−1(3)研究計画調書(添付ファイル項目)}\\
				\rule{0pt}{0pt}\\}
	\fancyhead[R]{\headerfont{\textbf{特別推進研究2−10}\\}}
}

\fancypagestyle{tokusui-2-11-header}{%
	\fancyhf{}
	\fancyhead[R]{\headerfont{\textbf{特別推進研究2−11}\\}}
}

\fancypagestyle{tokusui-2-12-header}{%
	\fancyhf{}
	\fancyhead[R]{\headerfont{\textbf{特別推進研究2−12}\\}}
}

%%\newcommand{\KLShumokuFirstPageStyle}[4]{%
%%%	Defines the header for the first page.
%%%	Called from \KLBeginSubject.
%%%--------------------------------
%%%	#1: page style name
%%%	#2: 様式
%%%	#3: 研究種目名
%%%	#4: 項目名
%%%--------------------------------
%%	\thispagestyle{#1}
%%}
%%
%%\newcommand{\KLShumokuDefaultPageStyle}[4]{%
%%%	Defines the default header.
%%%	Called from \KLBeginSubject.
%%%--------------------------------
%%%	#1: page style name
%%%	#2: 様式
%%%	#3: 研究種目名
%%%	#4: 項目名
%%%--------------------------------
%%	\pagestyle{#1}
%%}


%==========================================================
