%inst_kiban_chousenteki_houga.tex
\newcommand{\JSPSInstructions}{%
	\KLInstLine{}
	\KLInstructionTitle
	\KLInstLine{1}
	\begin{small}%
	\noindent
  1.本研究種目は、これまでの学術の体系や方向を大きく変革・転換させる潜在性を有する挑戦\\
    的研究を募集するものです((萌芽)については、探索的性質の強い、あるいは芽生え期の\\
    研究計画も対象としています)。応募に当たっては自身の研究計画がその趣旨に沿ったもの\\
    であるかを十分に確認すること。\\
  2.挑戦的研究(萌芽)は審査区分表の中区分により、広い分野の委員構成で多角的視点から審\\
    査が行われることに留意の上、研究計画調書を作成すること。\\
  3.挑戦的研究(萌芽)では、様式S--42--1(「研究計画調書の概要」欄)に研究計画調書\\
    (Web入力項目)の前半部分を加えた「研究計画調書(概要版)」のみによる事前の選考を\\
    行います(応募件数が少ない場合、事前の選考は行いません)。\\
  4.書面審査では、様式S--42--1(「研究計画調書の概要」欄)は参照できないため、\underline{様式}\\
    \underline{S--42--1(「研究計画調書の概要」欄)と本様式は独立に作成する必要があります。}例\\
    えば、様式S--42--1(「研究計画調書の概要」欄)に載せた図を本様式で引用すること\\
    はできないため、必要な図はそれぞれに記載すること。\\
	\end{small}
	
	\KLInstLine{2}
	\GeneralInstructions{}{}{}
}
