%#Split: 01_purpose_plan  
%#PieceName: p01_purpose_plan
% p01_purpose_plan_00.tex
\KLBeginSubject{01}{1}{1. Research Objectives, Research Method, etc.}{5}{F}{}{jsps-p1-header}{jsps-default-header}

\section{1 研究目的、研究方法など}
%    <<最大 6ページ>>

%s02_purpose_plan_with_abstract
\noindent
\textbf{(概要)}\\
%begin 研究目的及び研究計画の概要空行付き ====================
	象の卵の研究の目的は...	

	象の卵の研究計画と方法は...
	\vspace*{10zw}	% (概要)と(本文)の間が10行程度になるよう、必要に応じて値を調整してください。	
%end 研究目的及び研究計画の概要空行付き ====================

\noindent
\rule{\linewidth}{1pt}\\
\noindent
\textbf{(本文)}
%begin 研究目的と研究計画	====================
\JSPSInstructions	% <-- 留意事項。これは消すか、コメントアウトしてください。

\textbf{\\     *** 以下は、あくまで例です。真似しないでください。 ***\\
     *** 本文はもちろん、節の切り方や論理の組み方は   ***\\
     *** ご自分の気に入ったスタイルで書いてください。  ***}

	 象の卵の研究目的は...

	象の卵の研究計画は...

	%blahblah
An elephant was born from a big egg.
That big egg was hatched by its mother elephant.
That mother elephant was born from another egg.
That big egg was hatched by its mother elephant.
That mother elephant was born from another egg.
That big egg was hatched by its mother elephant.
That mother elephant was born from another egg.
That big egg was hatched by its mother elephant.
That mother elephant was born from another egg.
That big egg was hatched by its mother elephant.
That mother elephant was born from another egg.
That big egg was hatched by its mother elephant.
That mother elephant was born from another egg.
That big egg was hatched by its mother elephant.
That mother elephant was born from another egg.
That big egg was hatched by its mother elephant.
That mother elephant was born from another egg.
That big egg was hatched by its mother elephant.
That mother elephant was born from another egg.
That big egg was hatched by its mother elephant.
That mother elephant was born from another egg.
That big egg was hatched by its mother elephant.
That mother elephant was born from another egg.
That big egg was hatched by its mother elephant.
That mother elephant was born from another egg.
That big egg was hatched by its mother elephant.
That mother elephant was born from another egg.
That big egg was hatched by its mother elephant.
That mother elephant was born from another egg.
That big egg was hatched by its mother elephant.
That mother elephant was born from another egg.
That big egg was hatched by its mother elephant.
That mother elephant was born from another egg.
That big egg was hatched by its mother elephant.
That mother elephant was born from another egg.
That big egg was hatched by its mother elephant.
That mother elephant was born from another egg.
That big egg was hatched by its mother elephant.
That mother elephant was born from another egg.
That big egg was hatched by its mother elephant.
That mother elephant was born from another egg.
That big egg was hatched by its mother elephant.
That mother elephant was born from another egg.
That big egg was hatched by its mother elephant.
That mother elephant was born from another egg.
That big egg was hatched by its mother elephant.
That mother elephant was born from another egg.
That big egg was hatched by its mother elephant.
That mother elephant was born from another egg.
That big egg was hatched by its mother elephant.
That mother elephant was born from another egg.
That big egg was hatched by its mother elephant.
That mother elephant was born from another egg.
That big egg was hatched by its mother elephant.
That mother elephant was born from another egg.
That big egg was hatched by its mother elephant.
That mother elephant was born from another egg.
That big egg was hatched by its mother elephant.
That mother elephant was born from another egg.
That big egg was hatched by its mother elephant.
That mother elephant was born from another egg.
That big egg was hatched by its mother elephant.
That mother elephant was born from another egg.
That big egg was hatched by its mother elephant.
That mother elephant was born from another egg.
That big egg was hatched by its mother elephant.
That mother elephant was born from another egg.
That big egg was hatched by its mother elephant.
That mother elephant was born from another egg.
That big egg was hatched by its mother elephant.
  % << only for demonstration. Please delete it or comment it out.
	%blahblah
An elephant was born from a big egg.
That big egg was hatched by its mother elephant.
That mother elephant was born from another egg.
That big egg was hatched by its mother elephant.
That mother elephant was born from another egg.
That big egg was hatched by its mother elephant.
That mother elephant was born from another egg.
That big egg was hatched by its mother elephant.
That mother elephant was born from another egg.
That big egg was hatched by its mother elephant.
That mother elephant was born from another egg.
That big egg was hatched by its mother elephant.
That mother elephant was born from another egg.
That big egg was hatched by its mother elephant.
That mother elephant was born from another egg.
That big egg was hatched by its mother elephant.
That mother elephant was born from another egg.
That big egg was hatched by its mother elephant.
That mother elephant was born from another egg.
That big egg was hatched by its mother elephant.
That mother elephant was born from another egg.
That big egg was hatched by its mother elephant.
That mother elephant was born from another egg.
That big egg was hatched by its mother elephant.
That mother elephant was born from another egg.
That big egg was hatched by its mother elephant.
That mother elephant was born from another egg.
That big egg was hatched by its mother elephant.
That mother elephant was born from another egg.
That big egg was hatched by its mother elephant.
That mother elephant was born from another egg.
That big egg was hatched by its mother elephant.
That mother elephant was born from another egg.
That big egg was hatched by its mother elephant.
That mother elephant was born from another egg.
That big egg was hatched by its mother elephant.
That mother elephant was born from another egg.
That big egg was hatched by its mother elephant.
That mother elephant was born from another egg.
That big egg was hatched by its mother elephant.
That mother elephant was born from another egg.
That big egg was hatched by its mother elephant.
That mother elephant was born from another egg.
That big egg was hatched by its mother elephant.
That mother elephant was born from another egg.
That big egg was hatched by its mother elephant.
That mother elephant was born from another egg.
That big egg was hatched by its mother elephant.
That mother elephant was born from another egg.
That big egg was hatched by its mother elephant.
That mother elephant was born from another egg.
That big egg was hatched by its mother elephant.
That mother elephant was born from another egg.
That big egg was hatched by its mother elephant.
That mother elephant was born from another egg.
That big egg was hatched by its mother elephant.
That mother elephant was born from another egg.
That big egg was hatched by its mother elephant.
That mother elephant was born from another egg.
That big egg was hatched by its mother elephant.
That mother elephant was born from another egg.
That big egg was hatched by its mother elephant.
That mother elephant was born from another egg.
That big egg was hatched by its mother elephant.
That mother elephant was born from another egg.
That big egg was hatched by its mother elephant.
That mother elephant was born from another egg.
That big egg was hatched by its mother elephant.
That mother elephant was born from another egg.
That big egg was hatched by its mother elephant.
  % << only for demonstration. Please delete it or comment it out.
		

	\vspace{1cm}
	\begin{thebibliography}{99}
		\bibitem{teramura} 寺村輝夫、「ぼくは王様 - ぞうのたまごのたまごやき」.
	\end{thebibliography}
%end 研究目的と研究計画	====================

% p01_purpose_plan_01.tex
\KLEndSubject{F}


