%inst_kiban_chousenteki_kaitaku.tex
\newcommand{\JSPSInstructions}{%
	\KLInstLine{}
	\KLInstructionTitle
	\KLInstLine{1}
	\begin{small}%
	1. This research category calls for a challenging research with the potential of radically 	transforming the existing research framework and/or changing the research direction. Make sure that your research plan is consistent with the purpose of the research category.\\
	2. Proposals submitted to the research category Challenging Research (Pioneering) will be reviewed in the pertaining Medium-sized Section of the Review Section Table. The proposal document should be prepared with consideration that it will be reviewed from diverse viewpoints by a review committee consisting of reviewers with different backgrounds.\\
	3. In the research category Challenging Research (Pioneering), the preliminary screening will be conducted only by the ``Research Proposal Document (Outline)'' which is made up by adding the first half of the Research Proposal Document (items to be entered in the
	Website) to the form S-41-1 (``Outline of Research Proposal Document'' column) (Preliminary screening will not be conducted if the number of application is small).\\
	4. \underline{It is necessary to prepare the form S-41-1 (``Outline of Research Proposal Document'' column) and this form separately} since the form S-41-1 (``Outline of Research Proposal Document'' column) is unable to be referred in the document review and the panel review. For example, the necessary figures should be drawn on each form separately since the figures on the form S-41-1 (``Outline of Research Proposal Document'' column) are unable to be cited on this form. 
	\end{small}

	\KLInstLine{2}
	\GeneralInstructions{}{}{}
%	\DeleteInstructions{JSPSInstructions}
}

\renewcommand{\PapersInstructions}{%
	\KLInstLine{}
1. The description in this column does not require you to describe the details of research achievements (research papers, books, patents, invited talks, etc.) exhaustively. On citing papers etc. as necessary, for example, in the case of an article, fill in the information such as the title of the paper, the author(s), the title and the volume of the journal, the publication year, and the pages of the article so that the reviewers can identify the paper. \\
	\DeleteInstructions{PapersInstructions}\\
  ーーーーーーーーーーーーーーーーーーーーーーーーーーーーーーーーーーーーーーーーーーー\\
}