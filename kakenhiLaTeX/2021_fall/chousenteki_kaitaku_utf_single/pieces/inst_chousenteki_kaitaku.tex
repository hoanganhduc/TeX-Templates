%inst_kiban_chousenteki_kaitaku.tex
\newcommand{\JSPSInstructions}{%
	\KLInstLine{}
	\KLInstructionTitle
	\KLInstLine{1}
	\begin{small}%
  1.本研究種目は、これまでの学術の体系や方向を大きく変革・転換させる潜在性を有する挑戦\\
    的研究を募集するものです。応募に当たっては自身の研究計画がその趣旨に沿ったものであ\\
    るかを十分に確認すること。\\
  2.挑戦的研究(開拓)は審査区分表の中区分により、広い分野の委員構成で多角的視点から審\\
    査が行われることに留意の上、研究計画調書を作成すること。\\
  3.挑戦的研究(開拓)では、様式S--41--1(「研究計画調書の概要」欄)に研究計画調書\\
    (Web入力項目)の前半部分を加えた「研究計画調書(概要版)」のみによる事前の選考を\\
    行います(応募件数が少ない場合、事前の選考は行いません)。\\
  4.書面審査及び合議審査では、様式S--41--1(「研究計画調書の概要」欄)は参照できな\\
    いため、\underline{様式S--41--1(「研究計画調書の概要」欄)と本様式は独立に作成する必要が}\\
    \underline{あります。}例えば、様式S--41--1(「研究計画調書の概要」欄)に載せた図を本様式で\\
    引用することはできないため、必要な図はそれぞれに記載すること。\\
	\end{small}

	\KLInstLine{2}
	\GeneralInstructions{}{}{}
%	\DeleteInstructions{JSPSInstructions}
}

\renewcommand{\PapersInstructions}{%
	\KLInstLine{}
1.本欄は、研究業績(論文、著書、産業財産権、招待講演等)の詳細を網羅的に記載すること\\
  を求めるものではない。必要に応じて論文等を挙げる場合には、例えば論文であれば、論文\\
  名、著者名、掲載誌名、巻号や頁等、発表年(西暦)、といった当該論文が同定できる情報を\\
  記入すること。\\
	\DeleteInstructions{PapersInstructions}\\
  ーーーーーーーーーーーーーーーーーーーーーーーーーーーーーーーーーーーーーーーーーーー\\
}