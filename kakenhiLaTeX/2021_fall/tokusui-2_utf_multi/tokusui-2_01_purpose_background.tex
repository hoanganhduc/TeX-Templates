%#Split: 01_purpose_background  
%#PieceName: p01_purpose_background
% p01_purpose_background_00.tex
\KLBeginSubject{01}{1}{研究目的、背景など}{7}{V}{1}{tokusui-2-1-p1-header}{tokusui-2-1-header}

\section{研究目的、背景など}
%    <<最大 7ページ>>

%s02_purpose_baclground_tokusui-2.tex
\noindent
\mbox{\textbf{(概要)}}\\
%		<<10行程度>>
%begin 研究目的や背景などの概要 ====================
	象の卵の研究の目的は...	
	象の卵の研究計画と方法は...
%end 研究目的や背景などの概要 ====================

\noindent
\rule{\linewidth}{1pt}\\
\noindent
\mbox{\textbf{(本文)}}
%begin 研究目的や背景など ====================
\JSPSInstructions	% <-- 留意事項。これは消すか、コメントアウトしてください。

\textbf{\\     *** 以下は、あくまで例です。真似しないでください。 ***\\
     *** 本文はもちろん、節の切り方や論理の組み方は   ***\\
     *** ご自分の気に入ったスタイルで書いてください。  ***}

	 象の卵の研究目的は...

	唯一無二。

%blahblah
An elephant was born from a big egg.
That big egg was hatched by its mother elephant.
That mother elephant was born from another egg.
That big egg was hatched by its mother elephant.
That mother elephant was born from another egg.
That big egg was hatched by its mother elephant.
That mother elephant was born from another egg.
That big egg was hatched by its mother elephant.
That mother elephant was born from another egg.
That big egg was hatched by its mother elephant.
That mother elephant was born from another egg.
That big egg was hatched by its mother elephant.
That mother elephant was born from another egg.
That big egg was hatched by its mother elephant.
That mother elephant was born from another egg.
That big egg was hatched by its mother elephant.
That mother elephant was born from another egg.
That big egg was hatched by its mother elephant.
That mother elephant was born from another egg.
That big egg was hatched by its mother elephant.
That mother elephant was born from another egg.
That big egg was hatched by its mother elephant.
That mother elephant was born from another egg.
That big egg was hatched by its mother elephant.
That mother elephant was born from another egg.
That big egg was hatched by its mother elephant.
That mother elephant was born from another egg.
That big egg was hatched by its mother elephant.
That mother elephant was born from another egg.
That big egg was hatched by its mother elephant.
That mother elephant was born from another egg.
That big egg was hatched by its mother elephant.
That mother elephant was born from another egg.
That big egg was hatched by its mother elephant.
That mother elephant was born from another egg.
That big egg was hatched by its mother elephant.
That mother elephant was born from another egg.
That big egg was hatched by its mother elephant.
That mother elephant was born from another egg.
That big egg was hatched by its mother elephant.
That mother elephant was born from another egg.
That big egg was hatched by its mother elephant.
That mother elephant was born from another egg.
That big egg was hatched by its mother elephant.
That mother elephant was born from another egg.
That big egg was hatched by its mother elephant.
That mother elephant was born from another egg.
That big egg was hatched by its mother elephant.
That mother elephant was born from another egg.
That big egg was hatched by its mother elephant.
That mother elephant was born from another egg.
That big egg was hatched by its mother elephant.
That mother elephant was born from another egg.
That big egg was hatched by its mother elephant.
That mother elephant was born from another egg.
That big egg was hatched by its mother elephant.
That mother elephant was born from another egg.
That big egg was hatched by its mother elephant.
That mother elephant was born from another egg.
That big egg was hatched by its mother elephant.
That mother elephant was born from another egg.
That big egg was hatched by its mother elephant.
That mother elephant was born from another egg.
That big egg was hatched by its mother elephant.
That mother elephant was born from another egg.
That big egg was hatched by its mother elephant.
That mother elephant was born from another egg.
That big egg was hatched by its mother elephant.
That mother elephant was born from another egg.
That big egg was hatched by its mother elephant.

%blahblah
An elephant was born from a big egg.
That big egg was hatched by its mother elephant.
That mother elephant was born from another egg.
That big egg was hatched by its mother elephant.
That mother elephant was born from another egg.
That big egg was hatched by its mother elephant.
That mother elephant was born from another egg.
That big egg was hatched by its mother elephant.
That mother elephant was born from another egg.
That big egg was hatched by its mother elephant.
That mother elephant was born from another egg.
That big egg was hatched by its mother elephant.
That mother elephant was born from another egg.
That big egg was hatched by its mother elephant.
That mother elephant was born from another egg.
That big egg was hatched by its mother elephant.
That mother elephant was born from another egg.
That big egg was hatched by its mother elephant.
That mother elephant was born from another egg.
That big egg was hatched by its mother elephant.
That mother elephant was born from another egg.
That big egg was hatched by its mother elephant.
That mother elephant was born from another egg.
That big egg was hatched by its mother elephant.
That mother elephant was born from another egg.
That big egg was hatched by its mother elephant.
That mother elephant was born from another egg.
That big egg was hatched by its mother elephant.
That mother elephant was born from another egg.
That big egg was hatched by its mother elephant.
That mother elephant was born from another egg.
That big egg was hatched by its mother elephant.
That mother elephant was born from another egg.
That big egg was hatched by its mother elephant.
That mother elephant was born from another egg.
That big egg was hatched by its mother elephant.
That mother elephant was born from another egg.
That big egg was hatched by its mother elephant.
That mother elephant was born from another egg.
That big egg was hatched by its mother elephant.
That mother elephant was born from another egg.
That big egg was hatched by its mother elephant.
That mother elephant was born from another egg.
That big egg was hatched by its mother elephant.
That mother elephant was born from another egg.
That big egg was hatched by its mother elephant.
That mother elephant was born from another egg.
That big egg was hatched by its mother elephant.
That mother elephant was born from another egg.
That big egg was hatched by its mother elephant.
That mother elephant was born from another egg.
That big egg was hatched by its mother elephant.
That mother elephant was born from another egg.
That big egg was hatched by its mother elephant.
That mother elephant was born from another egg.
That big egg was hatched by its mother elephant.
That mother elephant was born from another egg.
That big egg was hatched by its mother elephant.
That mother elephant was born from another egg.
That big egg was hatched by its mother elephant.
That mother elephant was born from another egg.
That big egg was hatched by its mother elephant.
That mother elephant was born from another egg.
That big egg was hatched by its mother elephant.
That mother elephant was born from another egg.
That big egg was hatched by its mother elephant.
That mother elephant was born from another egg.
That big egg was hatched by its mother elephant.
That mother elephant was born from another egg.
That big egg was hatched by its mother elephant.


	\vspace{1cm}
	\begin{thebibliography}{99}
		\bibitem{teramura} 寺村輝夫、「ぼくは王様 - ぞうのたまごのたまごやき」.
	\end{thebibliography}
%end 研究目的や背景など ====================
%====================================

% p01_purpose_background_01.tex
\KLEndSubject{V}


