%inst_kiban_tokusui-2.tex
\newcommand{\JSPSInstructions}{%
	\\
	\KLInstLine{}
	\KLInstructionTitle
%	\DeleteInstructions{JSPSInstructions}\\
	\KLInstLine{1}
  1. 特別推進研究は人文社会・理工・生物の「系」の区分により、広い分野の委員構成\\
    で多角的視点から審査が行われることに留意の上、研究計画調書を作成すること。\\
	\KLInstLine{2}
	\GeneralInstructions{}{}{}
}

\renewcommand{\PapersInstructions}{%
	\KLInstLine{}
\begin{small}%
 1.研究代表者(研究分担者がいる場合は研究分担者も同様)の研究発表論文や著書、講演等の\\
   研究業績については、本欄ではなく「RECENT RESEARCH ACTIVITIES I (Publications)」\\
   及び「RECENT RESEARCH ACTIVITIES II (Invited Lectures and Talks, Prizes, etc.)」に\\
   主要なものを記載すること。\\
 2.本欄において、これまでの研究活動で得られた成果を示すに当たり、特定の具体的な研究業績\\
   (論文、著書、産業財産権、招待講演等)を明示する特段の必要がある場合は、当該業績を\\
   同定するに十分な情報を記載すること。\\
   例として、学術論文の場合は論文名、著者名、掲載誌名、巻号や頁等、発表年(西暦)、著書の場合は\\
   その書誌情報、など。\\
 3.論文は、既に掲載されているもの又は掲載が確定しているものに限って記載すること。\\
 \DeleteInstructions{PapersInstructions}\\
\end{small}
 ーーーーーーーーーーーーーーーーーーーーーーーーーーーーーーーーーーーーーーーー\\
}