% form04_kiban_s_header.tex
% 2018-09-02 Taku: Revised headers for JFY2019 style with (sub page#).

\usepackage{watermark}

% ===== Global definitions for the Kakenhi form ======================
% 基本情報
\newcommand{\様式}{様式S−11}
\newcommand{\研究種目}{基盤研究}
\newcommand{\研究種目後半}{}
\newcommand{\研究種別}{(S)}
\newcommand{\研究種目header}{\研究種目\研究種別\研究種目後半}

\newcommand{\KLMainFile}{kiban\_s.tex}
\newcommand{\KLYoshiki}{kiban_s_header}

% ===== Headers =====================================
\fancypagestyle{kiban-s-pub-subject-header}{%
	\fancyhf{}
	\fancyhead[L]{\framebox{\headerfont 研究代表者・研究分担者ごとに研究者調書を作成・添付 }\\}
	\fancyhead[R]{\headerfont\textbf{\研究種目\研究種別\ 9−(\KLJInt{\thepage})\\}}
}

\fancypagestyle{kiban-s-pub-header}{%
	\fancyhf{}
	\fancyhead[L]{\headerfont\textbf{【研究者調書(研究代表者)(つづき)】}}
	\fancyhead[R]{\headerfont\textbf{\研究種目\研究種別\ 9−(\KLJInt{\thepage})\\}}
}

%==== specific commands for kiban_s ================
\newcommand{\研究者氏名}{}
\newcommand{\研究者氏名ふりがな}{}
\newcommand{\研究者生年月日の年}{}
\newcommand{\研究者生年月日の月}{}
\newcommand{\研究者生年月日の日}{}
\newcommand{\研究者年齢}{}
\newcommand{\研究者所属機関部局職}{}
\newcommand{\研究者学位}{}

\newcommand{\KLXi}{}
\newcommand{\KLYi}{}
\newcommand{\KLYii}{}
%============================================
