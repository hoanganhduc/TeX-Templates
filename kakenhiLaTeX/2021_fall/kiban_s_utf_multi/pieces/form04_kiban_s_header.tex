% form04_kiban_s_header.tex
% 2018-09-02 Taku: Revised headers for JFY2019 style with (sub page#).

\usepackage{watermark}

% ===== Global definitions for the Kakenhi form ======================
% 基本情報
\newcommand{\様式}{Form S-11:}
\newcommand{\研究種目}{Scientific Research}
\newcommand{\研究種目後半}{}
\newcommand{\研究種別}{ (S)}
\newcommand{\研究種目header}{\研究種目\研究種別\研究種目後半}

\newcommand{\KLMainFile}{kiban\_s.tex}
\newcommand{\KLYoshiki}{kiban_s_header}

% ===== Headers =====================================
\fancypagestyle{kiban-s-pub-subject-header}{%
	\fancyhf{}
	\fancyhead[L]{\framebox{\headerfont\parbox{0.65\textwidth}{To be filled in and attached to the Research Proposal Document for\\ each Principal Investigator (PI) and Co-Investigator(s) (Co-I(s))}}\\}
	\fancyhead[R]{\headerfont\textbf{\研究種目\研究種別\ 9-(\thepage)\\}}
}

\fancypagestyle{kiban-s-pub-header}{%
	\fancyhf{}
	\fancyhead[L]{\headerfont\textbf{[4. Proposal of the Researcher (PI) (continued from the previous page)]}}
	\fancyhead[R]{\headerfont\textbf{\研究種目\研究種別\ 9-(\thepage)\\}}
}

%==== specific commands for kiban_s ================
\newcommand{\研究者氏名}{}
\newcommand{\研究者氏名ふりがな}{}
\newcommand{\研究者生年月日の年}{}
\newcommand{\研究者生年月日の月}{}
\newcommand{\研究者生年月日の日}{}
\newcommand{\研究者年齢}{}
\newcommand{\研究者所属機関部局職}{}
\newcommand{\研究者学位}{}

\newcommand{\KLXi}{}
\newcommand{\KLYi}{}
\newcommand{\KLYii}{}
%============================================
