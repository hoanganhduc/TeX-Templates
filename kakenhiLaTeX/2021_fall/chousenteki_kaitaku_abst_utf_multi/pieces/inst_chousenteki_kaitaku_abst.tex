%inst_kiban_chousenteki_houga_abst.tex
\newcommand{\JSPSInstructions}{%
    \begin{small}
	\KLInstLine{}
	\KLInstructionTitle
	\KLInstLine{1}
  1.本研究種目は、これまでの学術の体系や方向を大きく変革・転換させる潜在性を有する挑戦\\
    的研究を募集するものです。応募に当たっては自身の研究計画がその趣旨に沿ったものであ\\
    るかを十分に確認すること。\\
  2.挑戦的研究(開拓)は審査区分表の中区分により、広い分野の委員構成で多角的視点から審\\
    査が行われることに留意の上、研究計画調書を作成すること。\\
  3.挑戦的研究(開拓)では、本様式(「研究計画調書の概要」欄)に研究計画調書(Web入力\\
    項目)の前半部分を加えた「研究計画調書(概要版)」のみによる事前の選考を行います\\
   (応募件数が少ない場合、事前の選考は行いません)。本様式は書面審査及び合議審査では\\
    参照できないため注意すること。\\

	\KLInstLine{2}
	\GeneralInstructions{}{}{}
    \end{small}
}
