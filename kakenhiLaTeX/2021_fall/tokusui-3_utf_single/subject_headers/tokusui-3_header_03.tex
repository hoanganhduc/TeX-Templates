\documentclass[8pt]{extarticle}
\usepackage{newtxtext,newtxmath} % Times New Roman
\usepackage[top=0pt, bottom=0pt]{geometry}
\setlength{\oddsidemargin}{-8pt}
\setlength{\evensidemargin}{-8pt}
\usepackage{tabularx}
\usepackage{mathptmx}
\usepackage{array}
%https://tex.stackexchange.com/questions/41758/how-can-i-reproduce-this-table-with-thick-lines
\makeatletter
\newcommand{\thickhline}{%
	\noalign {\ifnum 0=`}\fi \hrule height 1pt
	\futurelet \reserved@a \@xhline
}
\newcolumntype{"}{@{\vrule width 1pt}}
\makeatother
\usepackage{graphicx}

\begin{document}	
\noindent\textbf{\fontsize{12}{12}\selectfont Reason(s) Why Comments by an Overseas Researcher is not Appropriate}\\
\begin{tabularx}{1.1\linewidth}{"|X|"}
	\thickhline
	As a rule, review comments from overseas researchers are solicited for a proposal submitted to the category of Specially Promoted Research. However, if the applicant considers that his/her proposal document should not be sent to overseas researchers for some reason, he/she can select ``Not Appropriate'' for the entry ``Appropriateness of Comments by Overseas Researchers'' (one of the web entry items). In that case, the applicant should give the reason of the choice within 1 page. The appropriateness of the reason of the choice will be judged by the Scientific Research Grant Committee.
	\\
	\thickhline
\end{tabularx}
\end{document}

