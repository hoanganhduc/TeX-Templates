%\documentclass[11pt,a4paper,uplatex,dvipdfmx]{ujarticle} 		% for uplatex
\documentclass[11pt,a4j,dvipdfmx]{jarticle} 					% for platex
\input{pieces/form00_header} % pieces
\input{pieces/kakenhi7} % pieces
% form01_header.tex
% 2017-05-28 Split from form00_header.tex to move \input{kakenhiLaTeX7.sty} to mother_1.tex.
% 2010-01-15 Adjusted margins.
% ===== Parameters for KL (Kakenhi LaTeX) ========================
%%\geometry{noheadfoot,scale=1}  %scale=1 resets margins to 0
\setlength{\unitlength}{1pt}

\newlength{\KLCella}
\newlength{\KLCellb}
\newlength{\KLCellc}
\newlength{\KLCelld}
\newlength{\KLCelle}
\newlength{\KLCellf}

\newcounter{KLMaxYearCount}	% # of years for the proposal
\newcommand{\KLCLLang}{}	% language-dependent left-justification in tabular

% ===== format and header =========
% 2020-01-15: Reset it to match the margins (25 mm on sides, 20 mm on top and bottom). 
% A4: 294 mm x 210 mm.  
% LaTeX's default margin is 1 inch = 25.4 mm.
\setlength{\oddsidemargin}{-1pt}	% (25.0 - 25.4) / 25.4 * 72 pt/inch = -1 pt
\setlength{\evensidemargin}{-1pt}
\setlength{\textwidth}{453pt}		% (210 - 25*2) / 25.4 * 72 = 453 pt
\setlength{\topmargin}{-61pt}	% This and \headheight determine the actual top margin
\setlength{\textheight}{254mm}		% (294 - 20*2) = 254 mm

\setlength{\headheight}{48pt}
\setlength{\headsep}{3pt}

\cfoot{}
\renewcommand{\headrulewidth}{0pt}

\pagestyle{empty}
% ==== other applications table =========
\newcommand{\KLTableHeaderFont}{\fontsize{8.2}{11}\selectfont}
\newcommand{\KLTableHeaderSmallFont}{\fontsize{7.5}{10}\selectfont}
\newcommand{\KLTableHeaderSmallerFont}{\fontsize{7}{10}\selectfont}

 % pieces
% ===== Global year-dependent definitions for the Kakenhi form ===========
% 基本情報
\newcommand{\研究開始年度}{2022}
\newcommand{\研究開始元号年度}{04}	%令和

\newcommand{\一年目西暦}{2022}
\newcommand{\二年目西暦}{2023}
\newcommand{\三年目西暦}{2024}
\newcommand{\四年目西暦}{2025}
\newcommand{\五年目西暦}{2026}
\newcommand{\六年目西暦}{2027}

\newcommand{\一年目}{4}
\newcommand{\二年目}{5}
\newcommand{\三年目}{6}
\newcommand{\四年目}{7}
\newcommand{\五年目}{8}
\newcommand{\六年目}{9}

\newcommand{\一年目J}{4}
\newcommand{\二年目J}{5}
\newcommand{\三年目J}{6}
\newcommand{\四年目J}{7}
\newcommand{\五年目J}{8}
\newcommand{\六年目J}{9}


 % pieces
\input{pieces/hook3} % pieces
%#Name: tokusui-1
% form04_jsps_headers.tex
% 2017-08-20 Taku
% 2017-08-29 Taku
%			Added a check against jsps-abs-p1-header.
% 2017-09-02 Taku
%			Added sectionNo to the commands to make them compatible with 
%			\KLBeginSubjectWithHeaderCommands.
%			Use \KLJInt.
% 2018-09-01 Taku
%			Adjusted the heights of the headers by inserting \vspace{-3pt} and \rule.
%
\newcommand{\headerfont}{\fontsize{11}{11}\selectfont}
% ===== Headers =====================================
\newcommand{\JSPSVeryFirstPageStyle}[5]{%
%	Defines the header for the very first page of the form.
%	Called from \KLShumokuFirstPageStyle in form04_***.
%--------------------------------
%	#1: page style name
%	#2: 様式
%	#3: 研究種目名
%	#4: 項目名
%	#5: sectionNo
%--------------------------------
	\fancypagestyle{JSPSVeryFirstPageStyle}{% The name is not taken from #1, because 
		\fancyhf{}
		\fancyhead[L]{\hspace{-37pt}\headerfont#2\ 研究計画調書(添付ファイル項目)\\
				\rule{0pt}{18pt}\\}
%				\rule{0pt}{0pt}\\}
		\fancyhead[R]{\headerfont\textbf{#3\ \KLJInt{\thepage}}\vspace{-5pt}\\
			\rule{0pt}{0pt}\\}
%		\fancyhead[R]{\headerfont\textbf{#3\ \KLJInt{\thepage}\\}}
	}
	\thispagestyle{JSPSVeryFirstPageStyle}
}

\newcommand{\JSPSFirstSubjectPageStyle}[5]{%
%	Defines the header for the first page for the subject.
%	Called from \KLShumokuFirstPageStyle in form04_***.
%--------------------------------
%	#1: page style name
%	#2: 様式
%	#3: 研究種目名
%	#4: 項目名
%	#5: sectionNo
%--------------------------------
	\fancypagestyle{JSPSFirstSubjectPageStyle}{%
		\fancyhf{}
		\fancyhead[R]{\headerfont\textbf{#3\ \KLJInt{\thepage}}\vspace{-5pt}\\
			\rule{0pt}{0pt}\\}
%		\fancyhead[R]{\headerfont\textbf{#3\ \KLJInt{\thepage}\\}}
	}
	\thispagestyle{JSPSFirstSubjectPageStyle}
}

\newcommand{\JSPSDefaultPageStyle}[5]{%
%	Defines the default header for the subject.
%	Called from \KLShumokuDefaultPageStyle in form04_***.
%--------------------------------
%	#1: page style name
%	#2: 様式
%	#3: 研究種目名
%	#4: 項目名
%	#5: sectionNo
%--------------------------------
	\fancypagestyle{JSPSDefaultPageStyle}{%
		\fancyhf{}
		\fancyhead[L]{\headerfont\textbf{【#4(つづき)\ 】}\vspace{-7pt}\\}
		\fancyhead[R]{\headerfont\textbf{#3\ \KLJInt{\thepage}}\vspace{-5pt}\\
			\rule{0pt}{0pt}\\}
%		\fancyhead[R]{\headerfont\textbf{#3\ \KLJInt{\thepage}\\}}	
        }
        \pagestyle{JSPSDefaultPageStyle}
}

 % pieces
% form04_tokusui-1_header.tex

\usepackage{watermark}

% ===== Global definitions for the Kakenhi form ======================
% 基本情報
\newcommand{\様式}{様式S−1(1)}
\newcommand{\研究種目}{特別推進研究}
\newcommand{\研究種目後半}{ 新規}
\ifthenelse{\isundefined{\研究種別}}{%
	\newcommand{\研究種別}{}
}{}%
\newcommand{\研究種目header}{\研究種目}

\newcommand{\KLMainFile}{tokusui-1.tex}
\newcommand{\KLYoshiki}{tokusui-1_header}

%==== specific commands for kiban_s ================
\newcommand{\Name}{}
\newcommand{\DateOfBirth}{}
\newcommand{\Age}{}
\newcommand{\AffiliationPosition}{}
\newcommand{\Degree}{}

\newcommand{\KLXi}{}

% ===== Headers =====================================
\setlength{\headsep}{-2pt}

\fancypagestyle{tokusui-1-p1-header}{%
	\fancyhf{}
	\fancyhead[L]{\hspace{-34pt}\headerfont{ Form S-1 (1): Research Proposal Document (forms to be uploaded)}\\
				\vspace*{10pt}
				\rule{0pt}{0pt}\\}
	\fancyhead[R]{\headerfont{\textbf{Specially Promoted Research 1-1-(\thepage)}\\\vspace*{8pt}}}
}

\fancypagestyle{tokusui-1-default-header}{%
	\fancyhf{}
	\fancyhead[R]{\headerfont{\textbf{Specially Promoted Research 1-1-(\thepage)}\\\vspace*{5pt}}}
}

\fancypagestyle{tokusui-1-subject-header}{%
	\fancyhf{}
	\fancyhead[R]{\headerfont{\textbf{Specially Promoted Research 1-\thepage}\\\vspace*{5pt}}}
}
%==========================================================
 % pieces
% ===== Global definitions for the Kakenhi form ======================
% 基本情報
%
%------ 研究課題名  -------------------------------------------
\newcommand{\研究課題名}{象の卵}

%----- 研究機関名と研究代表者の氏名-----------------------
\newcommand{\研究機関名}{逢坂大学}
\newcommand{\研究代表者氏名}{湯川秀樹}
\newcommand{\me}{\underline{\underline{H.~Yukawa}}} 
%---- 研究期間の最終年度 ----------------
\newcommand{\研究期間の最終元号年度}{8}  %令和で、半角数字のみ
%========================================

% inst_general.tex
%--------------------------------------------------------------------
% For writing instructions
%--------------------------------------------------------------------
\newcommand{\KLInstWOGeneral}[1]{%
	\noindent
 ーー Note ーーーーーーーーーーーーーーーーーーーーーーーーーーーーーー\\
		#1\\
 ーーーーーーーーーーーーーーーーーーーーーーーーーーーーーーーーーーーーーーーー
}

\newcommand{\KLInst}[1]{%
	\noindent
	\ifthenelse{\equal{#1}{}}{%
 ーー Note ーーーーーーーーーーーーーーーーーーーーーーーーーーーーーーー\\
	}{%
 ーー Note\textcircled{1} ーーーーーーーーーーーーーーーーーーーーーーーーーーーーーーー\\
		#1\\
		
	\noindent
 ーー Note\textcircled{2} ーーーーーーーーーーーーーーーーーーーーーーーーーーーーーーー\\
	}
}

\newcommand{\KLInstLine}[1]{%
	\ifthenelse{\equal{#1}{}}{%
 ーー Note ーーーーーーーーーーーーーーーーーーーーーーーーーーーーーー\\
	}{%
 ーー Note\textcircled{#1} ーーーーーーーーーーーーーーーーーーーーーーーーーーーーーーー\\
	}
}

\newcommand{\KLInstructionTitle}{%
	\textbf{\large Matters to be noted when preparing the Research Proposal Document}\\
}

\newcommand{\DeleteInstructions}[1]{%
	\textcolor{red}{Read the following important notes carefully before preparing this form. Delete this entire text box when filling in this form.\\
 (Delete \texttt{\textbackslash #1} and other text)}%
}

% local variables for \GeneralInstructions ------------------------
\newcommand{\留意事項内種目名}{}			% #1
\newcommand{\留意事項内記入要領名}{}		% #2
\newcommand{\留意事項内種目名削除用}{}	% #3

\newcommand{\KakenhiInstructions}{%
    \textbf{1.以下の内容を熟読・理解の上、研究計画調書を作成すること。\\
  ーーーーーーーーーーーーーーーーーーーーーーーーーーーーーーーーーーーーーーーーーーー\\
    \begin{small}
     科研費は、研究者の自由な発想に基づく全ての分野にわたる研究を格段に発展させることを目的とし、\\
    豊かな社会発展の基盤となる独創的・先駆的な研究を支援します。\\
     科研費では、応募者が自ら自由に課題設定を行うため、提案課題の学術的意義に加え、独自性や創造性\\
    が重要な評価ポイントになります。このため、「基盤研究」及び「若手研究」の研究計画調書様式では、\\
    学術の潮流や新たな展開などどのような「学術的背景」の下でどのような「学術的『問い』」を設定した\\
    か、当該課題の「学術的独自性や創造性」、「着想に至った経緯」、「国内外の研究動向と本研究の位置\\
    付け」はどのようなものか、などの記述を求めています。\\
     審査においては、総合審査又は二段階書面審査における審査委員間の議論・意見交換等により研究課題\\
    の核心を掴み、学術的な意義や独自性、創造性など学術的重要性を評価するとともに、実行可能性並びに\\
    研究遂行能力も含めて総合的に判断します。\\
     科研費に応募するに当たっては、上記に留意の上、公募要領や審査基準、様式の説明書き等を十分に確\\
    認し、審査委員に学術的重要性等が適切に伝わるように研究計画調書を作成してください。\\
    \end{small}%
  ーーーーーーーーーーーーーーーーーーーーーーーーーーーーーーーーーーーーーーーーーーー\\
    }
}

\newcommand{\GeneralInstructions}[3]{%
	\ifthenelse{\equal{#1}{}}{%
		\renewcommand{\留意事項内種目名}{研究計画調書}%
	}{%
		\renewcommand{\留意事項内種目名}{#1}%
	}%
	\ifthenelse{\equal{#2}{}}{%
		\renewcommand{\留意事項内記入要領名}{作成・記入要領}%
	}{%
		\renewcommand{\留意事項内記入要領名}{#2}%
	}%
	\ifthenelse{\equal{#3}{}}{%
		\renewcommand{\留意事項内種目名削除用}{\留意事項内種目名}%
	}{%
		\renewcommand{\留意事項内種目名削除用}{#3}%
	}%
	1. Read carefully the ``Procedures for Preparing and Entering a Research Proposal Document'' when preparing the document.\\
	2. The document should be written with font size 10-point or larger.\\
	3. The title and instructions on the upper part of each page should be left intact.\\
	4. Do not exceed the maximum number of pages specified in the instructions. In case blank page(s) occur, leave them as they are (do not eliminate any page).\\ 
 \DeleteInstructions{JSPSInstructions}\\
 ーーーーーーーーーーーーーーーーーーーーーーーーーーーーーーーーーーーーーーーー
}


\newcommand{\PapersInstructions}{%
 ーー Note ーーーーーーーーーーーーーーーーーーーーーーーーーーーーーーー\\
\begin{small}%
 1. 研究業績(論文、著書、産業財産権、招待講演等)は、網羅的に記載するのではなく、本研\\
   究計画の実行可能性を説明する上で、その根拠となる文献等の主要なものを適宜記載するこ\\
   と。\\
 2.研究業績の記述に当たっては、当該研究業績を同定するに十分な情報を記載すること。\\
   例として、学術論文の場合は論文名、著者名、掲載誌名、巻号や頁等、発表年(西暦)、著\\
   書の場合はその書誌情報、など。\\
 3.論文は、既に掲載されているもの又は掲載が確定しているものに限って記載すること。\\
 \DeleteInstructions{PapersInstructions}\\
\end{small}
 ーーーーーーーーーーーーーーーーーーーーーーーーーーーーーーーーーーーーーーーー\\
} % pieces
%inst_kiban_tokusui-1.tex
\newcommand{\JSPSInstructions}{%
	\KLInstWOGeneral{
  1.作成に当たっては、研究計画調書作成・入力要領を必ず確認すること。\\
  2.このファイルについては、記入は全て英語で行うこと。\\
  3.使用する文字サイズは、10ポイント以上とすること。\\
  4.各頁の上部のタイトルと指示書きは動かさないこと。\\
  5.指示書きで定められた頁数は超えないこと。なお、空白の頁が生じても削除しないこと。\\
  6.\textcolor{red}{本留意事項の内容を十分に確認し、研究計画調書の作成時には\\
    本留意事項を削除すること。(\texttt{\textbackslash JSPSInstructions}を消す)}
	}
}

\renewcommand{\PapersInstructions}{%
	\KLInstWOGeneral{%
    {\it
    \noindent
    List the significant academic contributions (research papers, articles, books) and intellectual properties (patents). Achievement not directly related to this proposed project can be included. Begin with the most recent one. Do not include research papers under submission. Textbooks, abstracts for conferences and address summaries should not be included in this list either.\\
    
    \noindent
    Title and Authors etc.\\
    (e.g., For research papers, list the title of the paper, authors, name of the journal, refereed\\
     or not, volume number, the first and last page numbers, year of publication)\\
    
    \noindent
    Notes:\\
    1. It is not necessary for above information to be listed in this order shown above, as long \\
    \hspace{5mm}    as all information is included.\\
    2. You need not list up all co-authors.
    
%    \textcolor{red}{The parts in italics should be deleted when filling this column.\\
    \textcolor{red}{本留意事項の内容を十分に確認し、研究計画調書の作成時には\\
 本留意事項を削除すること。
	(Delete \texttt{\textbackslash PapersInstructions}.)}
    }
    }
}

\newcommand{\TalksInstructions}{%
	\KLInstWOGeneral{%
    {\it
    	\noindent
    	List the important lectures/talks (e.g., invited lecture at an international conference) and prizes.\\
	Name of Conference, Date and Place, 
	Title of Lecture(s)/Talk(s), Name of Prizes. \\
	Begin with the most recent one.\\
	
%    \textcolor{red}{The parts in italics should be deleted when filling this column.\\
    \textcolor{red}{本留意事項の内容を十分に確認し、研究計画調書の作成時には\\
 本留意事項を削除すること。
	(Delete \texttt{\textbackslash TalksInstructions}.)}
    }
    }
} % pieces
% user07_header
% ===== my favorite packages ====================================
% ここに、自分の使いたいパッケージを宣言して下さい。
\usepackage{wrapfig}
%\usepackage{amssymb}
%\usepackage{mb}
%\DeclareGraphicsRule{.tif}{png}{.png}{`convert #1 `dirname #1`/`basename #1 .tif`.png}
\usepackage{lineno}

% ===== my personal definitions ==================================
% ここに、自分のよく使う記号などを定義して下さい。
\newcommand{\klpionn}{K_L \to \pi^0 \nu \overline{\nu}}
\newcommand{\kppipnn}{K^+ \to \pi^+ \nu \overline{\nu}}

% ----- 業績リスト用 -------------
\newcommand{\paper}[6]{%
	% paper{title}{authors}{journal}{vol}{pages}{year}
	\item ``#1'', #2, #3 {\bf #4}, #5 (#6).			% お好みに合わせて変えてください。
}

\newcommand{\etal}{\textit{et al.\ }}
\newcommand{\ca}[1]{*#1}	% corresponding author;   \ca{\yukawa}  みたいにして使う
\newcommand{\invitedtalk}{招待講演}

\newcommand{\yukawa}{H.~Yukawa}					% no underline
%\newcommand{\yukawa}{\underline{\underline{H.~Yukawa}}}	% with 2 underlines
\newcommand{\tomonaga}{S.~Tomonaga}

\newcommand{\prl}{Phys.\ Rev.\ Lett.\ }		% よく使う雑誌も定義すると楽

% ===== 欄外メモ ==================
\newcommand{\memo}[1]{\marginpar{#1}}
%\renewcommand{\memo}[1]{}	% 全てのメモを表示させないようにするには、行頭の"%"を消す


%\input{../../sample/simple/contents}	% skip
\input{pieces/hook5} % pieces

\begin{document}
\input{pieces/hook7} % pieces
%#Split: 01_project_description  
%#PieceName: p01_project_description
% p01_project_description_00.tex
\KLBeginSubject{01}{1}{Project Description}{4}{V}{}{tokusui-1-p1-header}{tokusui-1-default-header}

\section{Project Description}
%    <<最大 4ページ>>

%s02_project_description
%begin projectDescription ====================
	\vspace*{-12mm}
\subsection*{Abstract}
	We will search for elephant eggs.
	
\subsection*{(1) Background of the Research Project}
	We have been searching for whale eggs for many years.
	
\subsection*{(2) Research Objectives and Targeted Goals of Project}
	Our goal is to find elephant eggs.

\subsection*{(3) Research Plan and Method}
	In the first year, we will visit major zoos in the world.
	We will search for the eggs in Africa in the second year, 
	and in India in the third year.
	
\subsection*{(4) Importance and Necessity of this Project and its Expected Impact on Broader Research Fields}
	A discovery of an elephant egg will completely change the concept of mammals.

\subsection*{(5) Research Achievements of the Applicant(s) Relevant to this Project}
	The applicant has been searching for whale eggs.
	
\JSPSInstructions	% <-- 留意事項。これは消すか、コメントアウトしてください。

%blahblah
An elephant was born from a big egg.
That big egg was hatched by its mother elephant.
That mother elephant was born from another egg.
That big egg was hatched by its mother elephant.
That mother elephant was born from another egg.
That big egg was hatched by its mother elephant.
That mother elephant was born from another egg.
That big egg was hatched by its mother elephant.
That mother elephant was born from another egg.
That big egg was hatched by its mother elephant.
That mother elephant was born from another egg.
That big egg was hatched by its mother elephant.
That mother elephant was born from another egg.
That big egg was hatched by its mother elephant.
That mother elephant was born from another egg.
That big egg was hatched by its mother elephant.
That mother elephant was born from another egg.
That big egg was hatched by its mother elephant.
That mother elephant was born from another egg.
That big egg was hatched by its mother elephant.
That mother elephant was born from another egg.
That big egg was hatched by its mother elephant.
That mother elephant was born from another egg.
That big egg was hatched by its mother elephant.
That mother elephant was born from another egg.
That big egg was hatched by its mother elephant.
That mother elephant was born from another egg.
That big egg was hatched by its mother elephant.
That mother elephant was born from another egg.
That big egg was hatched by its mother elephant.
That mother elephant was born from another egg.
That big egg was hatched by its mother elephant.
That mother elephant was born from another egg.
That big egg was hatched by its mother elephant.
That mother elephant was born from another egg.
That big egg was hatched by its mother elephant.
That mother elephant was born from another egg.
That big egg was hatched by its mother elephant.
That mother elephant was born from another egg.
That big egg was hatched by its mother elephant.
That mother elephant was born from another egg.
That big egg was hatched by its mother elephant.
That mother elephant was born from another egg.
That big egg was hatched by its mother elephant.
That mother elephant was born from another egg.
That big egg was hatched by its mother elephant.
That mother elephant was born from another egg.
That big egg was hatched by its mother elephant.
That mother elephant was born from another egg.
That big egg was hatched by its mother elephant.
That mother elephant was born from another egg.
That big egg was hatched by its mother elephant.
That mother elephant was born from another egg.
That big egg was hatched by its mother elephant.
That mother elephant was born from another egg.
That big egg was hatched by its mother elephant.
That mother elephant was born from another egg.
That big egg was hatched by its mother elephant.
That mother elephant was born from another egg.
That big egg was hatched by its mother elephant.
That mother elephant was born from another egg.
That big egg was hatched by its mother elephant.
That mother elephant was born from another egg.
That big egg was hatched by its mother elephant.
That mother elephant was born from another egg.
That big egg was hatched by its mother elephant.
That mother elephant was born from another egg.
That big egg was hatched by its mother elephant.
That mother elephant was born from another egg.
That big egg was hatched by its mother elephant.

%end projectDescription ====================
% p01_project_description_01.tex
\KLEndSubject{V}


%#Split: members_info % keep

% Principal Investigator ===============
%#Split: 02_cv % extract 
%#PieceName: p02_cv
\input{pieces/p02_cv_00}
\section{Curricula Vitae}
%    <<最大 1ページ>>

%s13_cv_en.tex
%s13_cv_en_commands
%begin CV info ===========================
\renewcommand{\Name}{Hideki YUKAWA}
\renewcommand{\DateOfBirth}{Feb. 29, 1900}
\renewcommand{\Age}{139}
\renewcommand{\AffiliationPosition}{\small{Ausaka University, Shell Lab., Professor Emeritus}}
\renewcommand{\Degree}{Ph.~D}
%end CV info ============================ 
%s13_cv_en_picture
\renewcommand{\KLXi}{200}

\thiswatermark{%
	\begin{picture}(0,0)(0,0)
		\put(\KLXi, -27){\Name}
		\put(\KLXi, -45){\DateOfBirth}
		\put(380, -45){\Age}
		\put(\KLXi, -61){%
			\parbox[t][26pt]{260pt}{\AffiliationPosition} 
		}
		\put(\KLXi, -120){\Degree}
	\end{picture}
}
 % pieces

%begin cvEnPi ====================
\vspace*{-8mm}
\subsection*{2. Roles in this Project}
	Spokesperson of the research

\subsection*{3. Research Career and Experience}
	I have travelled all around the world and 
	became fascinated with large mammals like whales and elephants.
%end cvEnPi ====================

% p02_cv_01.tex
\KLEndSubject{F}


%#Split: 03_publications % extract 
%#PieceName: p03_publications
% p03_publications_00.tex
\KLBeginSubject{03}{3}{Publications}{1}{F}{}{tokusui-1-subject-header}{jsps-default-header}

\section{Publications}
%    <<最大 1ページ>>

% s14_pub_en
%s14_pub_en_picture
\thiswatermark{%
	\begin{picture}(0,0)(0,0)
		\put(125, -28){\Name}
	\end{picture}
}
 % pieces

%begin publicationListPi ====================
\PapersInstructions	% <-- Notes。Please comment-out or delete this line.
	\begin{enumerate}
		\item \label{pub:theegg} +``Theory of Elephant Eggs'', 
				*\underline{Juzo Kara}, \me\ {\it et al.},
				Phys.\ Rev.\ Lett. {\bf 800}, 800-804 (2016). 

		\item \label{pub:theowhale} ``Theory of Whale Eggs'',
				*\underline{Juzo Kara}, \me\ {\it et al.},
				Phys.\ Rev.\ Lett. {\bf 800}, 805-808 (2016). 
	
		\item \label{pub:abe} ``Elephant's Child is Dead'', 
				*\underline{Kobo Abe},
				The Complete Works of Kobo Abe , {\bf 26}, 100-200 (2015). 

		\item \label{pub:abe} ``The Elephant's Child'', 
				*\underline{R.~Kipling},
				Nature, {\bf 999}, 777-779 (2014). 

		\item ``You can't Lay an Egg If You're an Elephant'', 
				*\underline{F.~Ehrlich},
				JofUR\\
				 ({\tt www.universalrejection.org}), {\bf N/A}, N/A (2013). 
		\item +``Egg of Elephant-Bird'', 
				*\underline{A.~Cooper},
				Nature, {\bf 409}, 704-707 (2012). 
	\end{enumerate}
%end publicationListPi ====================

% p03_publications_01.tex
\KLEndSubject{F}


%#Split: 04_talks % extract 
%#PieceName: p04_talks
% p04_talks_00.tex
\KLBeginSubject{04}{4}{Talks}{1}{F}{}{tokusui-1-subject-header}{jsps-default-header}

\section{Talks}
%    <<最大 1ページ>>

% s14_talks_en
%s14_talks_en_picture
\thiswatermark{%
	\begin{picture}(0,0)(0,0)
		\put(130, -32){\Name}
	\end{picture}
}
 % pieces

%begin talksListPi ====================
\TalksInstructions	% <-- Notes.  Please comment-out or delete this line.
	\begin{enumerate}
		\item + Hideki Yukawa, 
            		International Endeavor for Elephant's Egg (IEEE) Conference, 
            			Nov. 15, 2016 , Paris,
            		 ``Theory of Elephant's Eggs''.

		\item Richard Feynman, 
            		International Conference on Huge Elephant Physics (ICHEP2008),
            		   August 1-7, 2016 , Philadelphia, USA,
            		 ``Path Integral for Reaching Elephant's Eggs''.

		\item H.~Yukawa and Jacques-Yves Cousteau, 
            		Workshop on Oceanic Search,
            		April 1, 2015 , Hawaii, USA.
            		 ``How to search for whale eggs''.

		\item H.~Yukawa, 
            		Noel Prize, December 25, 2006, North Pole.
	\end{enumerate}
%end talksListPi ====================

% p04_talks_01.tex
\KLEndSubject{F}




% Co-investigators ===============
%#Split: 02_cv_b % extract 
%#PieceName: p02_cv_b
% p02_cv_b_00.tex
\KLBeginSubject{02}{2}{Curricula Vitae}{1}{F}{2}{tokusui-1-subject-header}{jsps-default-header}

\section{Curricula Vitae}
%    <<最大 1ページ>>

%s13_cv_en_b.tex
%s13_cv_en_commands_b
%begin CV info ===========================
\renewcommand{\Name}{Shinichiro TOMONAGA}
\renewcommand{\DateOfBirth}{Apr. 31, 1900}
\renewcommand{\Age}{137}
\renewcommand{\AffiliationPosition}{\small{Edo University, School of Science., Professor Emeritus}}
\renewcommand{\Degree}{Ph.~D}
%end CV info ============================ 
%s13_cv_en_picture
\renewcommand{\KLXi}{200}

\thiswatermark{%
	\begin{picture}(0,0)(0,0)
		\put(\KLXi, -27){\Name}
		\put(\KLXi, -45){\DateOfBirth}
		\put(380, -45){\Age}
		\put(\KLXi, -61){%
			\parbox[t][26pt]{260pt}{\AffiliationPosition} 
		}
		\put(\KLXi, -120){\Degree}
	\end{picture}
}
 % pieces

%begin cvEnCi ====================
\vspace*{-8mm}
\subsection*{2. Roles in this Project}
	Hypothetical model building

\subsection*{3. Research Career and Experience}
	I have been integrating the paths of every elephant in the world.
	The existence of eggs will make a perturbation on the calculated result.
%end cvEnCi ====================

% p02_cv_b_01.tex
\KLEndSubject{F}


%#Split: 03_publications_b % extract 
%#PieceName: p03_publications_b
\input{pieces/p03_publications_b_00}
\section{Publications}
%    <<最大 1ページ>>

% s14_pub_en_b
%s14_pub_en_picture
\thiswatermark{%
	\begin{picture}(0,0)(0,0)
		\put(125, -28){\Name}
	\end{picture}
}
 % pieces

%begin publicationListCi ====================
\PapersInstructions	% <-- Notes。Please comment-out or delete this line.
	\begin{enumerate}
		\item \label{pub:abe} ``The Elephant's Child'', 
				*\underline{R.~Kipling},
				Nature, {\bf 999}, 777-779 (2014). 

		\item ``You can't Lay an Egg If You're an Elephant'', 
				*\underline{F.~Ehrlich},
				JofUR\\
				 ({\tt www.universalrejection.org}), {\bf N/A}, N/A (2013). 
		\item +``Egg of Elephant-Bird'', 
				*\underline{A.~Cooper},
				Nature, {\bf 409}, 704-707 (2012). 
	\end{enumerate}
%end publicationListCi ====================

% p03_publications_b_01.tex
\KLEndSubject{F}


%#Split: 04_talks_b % extract 
%#PieceName: p04_talks_b
% p04_talks_b_00.tex
\KLBeginSubject{04}{4}{Talks}{1}{F}{}{tokusui-1-subject-header}{jsps-default-header}

\section{Talks}
%    <<最大 1ページ>>

% s14_talks_en_b
%s14_talks_en_picture
\thiswatermark{%
	\begin{picture}(0,0)(0,0)
		\put(130, -32){\Name}
	\end{picture}
}
 % pieces

%begin talksListCi ====================
\TalksInstructions	% <-- Notes.  Please comment-out or delete this line.
	\begin{enumerate}
		\item + \underline{S.~Tomonaga},
			International Conference on Perturbations,
			Nov. 31, 2017, London,
			``Solution to infinity large eggs''.
	\end{enumerate}
%end talksListCi ====================

% p04_talks_b_01.tex
\KLEndSubject{F}




% To add co-investigators, make a copy of info_coi.tex, rename it, and 
% add the file below.

%\input{info_taro}
%\input{info_hanako}
%\input{info_jiro}
%\input{info_saburo}


%#Split: 99_tail
\input{pieces/hook9} % pieces
\end{document}

