% inst_general.tex
%--------------------------------------------------------------------
% For writing instructions
%--------------------------------------------------------------------
\newcommand{\KLInstWOGeneral}[1]{%
	\noindent
 ーー Note ーーーーーーーーーーーーーーーーーーーーーーーーーーーーーー\\
		#1\\
 ーーーーーーーーーーーーーーーーーーーーーーーーーーーーーーーーーーーーーーーー
}

\newcommand{\KLInst}[1]{%
	\noindent
	\ifthenelse{\equal{#1}{}}{%
 ーー Note ーーーーーーーーーーーーーーーーーーーーーーーーーーーーーーー\\
	}{%
 ーー Note\textcircled{1} ーーーーーーーーーーーーーーーーーーーーーーーーーーーーーーー\\
		#1\\
		
	\noindent
 ーー Note\textcircled{2} ーーーーーーーーーーーーーーーーーーーーーーーーーーーーーーー\\
	}
}

\newcommand{\KLInstLine}[1]{%
	\ifthenelse{\equal{#1}{}}{%
 ーー Note ーーーーーーーーーーーーーーーーーーーーーーーーーーーーーー\\
	}{%
 ーー Note\textcircled{#1} ーーーーーーーーーーーーーーーーーーーーーーーーーーーーーーー\\
	}
}

\newcommand{\KLInstructionTitle}{%
	\textbf{\large Matters to be noted when preparing the Research Proposal Document}\\
}

\newcommand{\DeleteInstructions}[1]{%
	\textcolor{red}{Read the following important notes carefully before preparing this form. Delete this entire text box when filling in this form.\\
 (Delete \texttt{\textbackslash #1} and other text)}%
}

% local variables for \GeneralInstructions ------------------------
\newcommand{\留意事項内種目名}{}			% #1
\newcommand{\留意事項内記入要領名}{}		% #2
\newcommand{\留意事項内種目名削除用}{}	% #3

\newcommand{\KakenhiInstructions}{%
    \textbf{1. Read and understand the following important notes carefully before preparing your Research Proposal Document.\\
  ーーーーーーーーーーーーーーーーーーーーーーーーーーーーーーーーーーーーーーーーーーー\\
    \begin{small}
	KAKENHI funding aims to promote scientific research in all fields based on original ideas of researchers.
	The grants provide financial support for creative and pioneering research projects that will become the foundation of social development.
	In KAKENHI, research theme setting is at the applicant's discretion.
	As such, KAKENHI research proposals are evaluated based not only on their scientific significance, but also on their originality and creativity.
	Accordingly, in the Research Proposal Document forms for the ``Scientific Research'' and ``Early-Career Scientists'' categories, applicants are required to state:\\
	\checked\ What kind of \underline{key scientific question(s)} is set against the relevant \underline{scientific background} (such as research trends and new developments)?\\
	\checked\ What are the \underline{scientific originality} and \underline{creativity of the proposal}?\\
	\checked\ What was the \underline{research development leading to the conception of the research idea}?\\
	\checked\ What are \underline{the research trends (domestic and overseas) and the positioning of this research in the relevant field}?\\
	In the review process, research proposals will be screened either by Comprehensive Review or Two-Stage Document Review.
	Reviewers strive to grasp the essence of the proposed research through exchange of opinions among them, evaluate such merits as scientific significance, originality and creativity, and comprehensively place their judgments taking account of the feasibility of the research plan and the applicant's ability to conduct research.\\
	 	In applying for KAKENHI, applicants are advised to take note of the above, and to read the Application Procedures for Grants-in-Aid for Scientific Research and the explanations of review criteria and the annotations in the application form in preparing their Research Proposal Documents, so that the scientific merits and other points in the research proposal will be appropriately conveyed to the reviewers.\\
    \end{small}%
  ーーーーーーーーーーーーーーーーーーーーーーーーーーーーーーーーーーーーーーーーーーー\\
    }
}

\newcommand{\GeneralInstructions}[3]{%
	\ifthenelse{\equal{#1}{}}{%
		\renewcommand{\留意事項内種目名}{研究計画調書}%
	}{%
		\renewcommand{\留意事項内種目名}{#1}%
	}%
	\ifthenelse{\equal{#2}{}}{%
		\renewcommand{\留意事項内記入要領名}{作成・記入要領}%
	}{%
		\renewcommand{\留意事項内記入要領名}{#2}%
	}%
	\ifthenelse{\equal{#3}{}}{%
		\renewcommand{\留意事項内種目名削除用}{\留意事項内種目名}%
	}{%
		\renewcommand{\留意事項内種目名削除用}{#3}%
	}%
	1.	Read carefully the ``Procedures for Preparing and Entering a Research Proposal Document'' when preparing the document.\\
	2.	The document should be written with font size 10-point or larger.\\
	3.	The title and instructions on the upper part of each page should be left intact.\\
	4.	Do not exceed the maximum number of pages specified in the instructions. In case blank page(s) occur, leave them as they are (do not eliminate any page).\\
 \DeleteInstructions{JSPSInstructions}\\
 ーーーーーーーーーーーーーーーーーーーーーーーーーーーーーーーーーーーーーーーー
}


\newcommand{\PapersInstructions}{%
 ーー Note ーーーーーーーーーーーーーーーーーーーーーーーーーーーーーーー\\
\begin{small}%
1. The description in this column is to explain the feasibility of the research plan. On citing research achievements (research papers, books, patents, invited talks, etc.) they should be given not as an exhaustive list but as supporting evidence to prove the applicant's ability to conduct the proposed research.\\
2. Sufficient information should be given so that the reviewers can identify the research achievements. In the case of a research paper, for example, the relevant bibliographic information, including the title of the paper, the author(s), the title and the volume of the journal, the publication year, and the pages of the article should be given.\\
3. The research papers that can be cited are only those already published or accepted for publication.\\
 \DeleteInstructions{PapersInstructions}\\
\end{small}
 ーーーーーーーーーーーーーーーーーーーーーーーーーーーーーーーーーーーーーーーー\\
}
