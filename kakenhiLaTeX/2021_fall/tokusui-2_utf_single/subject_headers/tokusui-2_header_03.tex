\documentclass[8pt]{extarticle}
\usepackage{newtxtext,newtxmath} % Times New Roman
\usepackage[top=0pt, bottom=0pt]{geometry}
\setlength{\oddsidemargin}{-8pt}
\setlength{\evensidemargin}{-8pt}
\usepackage{tabularx}
\usepackage{mathptmx}
\usepackage{array}
%https://tex.stackexchange.com/questions/41758/how-can-i-reproduce-this-table-with-thick-lines
\makeatletter
\newcommand{\thickhline}{%
	\noalign {\ifnum 0=`}\fi \hrule height 1pt
	\futurelet \reserved@a \@xhline
}
\newcolumntype{"}{@{\vrule width 1pt}}
\makeatother

\begin{document}	
\noindent\textbf{\fontsize{12}{12}\selectfont Research Plan and Methods}\\
\begin{tabularx}{1.1\linewidth}{"|X|"}
	\thickhline
	The applicant should provide details of the research plan and methods for achieving the objectives of the research. The following points should be stated in concrete and clear terms.
	
	- Preparation status for the research plan: The description with an understandable manner should include a preparation status such as a data collection, analysis, assessment and examination, a preliminary experiment to become a foundation, a design and manufacture of the experimental devices or facilities, a development of methods, and an organization of the project members including a participation of Research Collaborator(s), etc., which is assumed to be a precondition for the implementation of the proposed research project and also explain on the relations between the preparation status and the research plan.
	
	- A concrete description of the role-sharing between the Principal Investigator (PI) and the Co-Investigator(s) (Co-I(s)), if the proposed research project involves Co-I(s).
	\\
	\thickhline
\end{tabularx}
\end{document}

