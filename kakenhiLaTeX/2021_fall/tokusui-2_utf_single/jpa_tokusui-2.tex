%\documentclass[11pt,a4paper,uplatex,dvipdfmx]{ujarticle} 		% for uplatex
\documentclass[11pt,a4j,dvipdfmx]{jarticle} 					% for platex
\input{pieces/form00_header} % pieces
\input{pieces/kakenhi7} % pieces
% form01_header.tex
% 2017-05-28 Split from form00_header.tex to move \input{kakenhiLaTeX7.sty} to mother_1.tex.
% 2010-01-15 Adjusted margins.
% ===== Parameters for KL (Kakenhi LaTeX) ========================
%%\geometry{noheadfoot,scale=1}  %scale=1 resets margins to 0
\setlength{\unitlength}{1pt}

\newlength{\KLCella}
\newlength{\KLCellb}
\newlength{\KLCellc}
\newlength{\KLCelld}
\newlength{\KLCelle}
\newlength{\KLCellf}

\newcounter{KLMaxYearCount}	% # of years for the proposal
\newcommand{\KLCLLang}{}	% language-dependent left-justification in tabular

% ===== format and header =========
% 2020-01-15: Reset it to match the margins (25 mm on sides, 20 mm on top and bottom). 
% A4: 294 mm x 210 mm.  
% LaTeX's default margin is 1 inch = 25.4 mm.
\setlength{\oddsidemargin}{-1pt}	% (25.0 - 25.4) / 25.4 * 72 pt/inch = -1 pt
\setlength{\evensidemargin}{-1pt}
\setlength{\textwidth}{453pt}		% (210 - 25*2) / 25.4 * 72 = 453 pt
\setlength{\topmargin}{-61pt}	% This and \headheight determine the actual top margin
\setlength{\textheight}{254mm}		% (294 - 20*2) = 254 mm

\setlength{\headheight}{48pt}
\setlength{\headsep}{3pt}

\cfoot{}
\renewcommand{\headrulewidth}{0pt}

\pagestyle{empty}
% ==== other applications table =========
\newcommand{\KLTableHeaderFont}{\fontsize{8.2}{11}\selectfont}
\newcommand{\KLTableHeaderSmallFont}{\fontsize{7.5}{10}\selectfont}
\newcommand{\KLTableHeaderSmallerFont}{\fontsize{7}{10}\selectfont}

 % pieces
% ===== Global year-dependent definitions for the Kakenhi form ===========
% 基本情報
\newcommand{\研究開始年度}{2022}
\newcommand{\研究開始元号年度}{04}	%令和

\newcommand{\一年目西暦}{2022}
\newcommand{\二年目西暦}{2023}
\newcommand{\三年目西暦}{2024}
\newcommand{\四年目西暦}{2025}
\newcommand{\五年目西暦}{2026}
\newcommand{\六年目西暦}{2027}

\newcommand{\一年目}{4}
\newcommand{\二年目}{5}
\newcommand{\三年目}{6}
\newcommand{\四年目}{7}
\newcommand{\五年目}{8}
\newcommand{\六年目}{9}

\newcommand{\一年目J}{4}
\newcommand{\二年目J}{5}
\newcommand{\三年目J}{6}
\newcommand{\四年目J}{7}
\newcommand{\五年目J}{8}
\newcommand{\六年目J}{9}


 % pieces
\input{pieces/hook3} % pieces
%#Name: tokusui-2
% form04_jsps_headers.tex
% 2017-08-20 Taku
% 2017-08-29 Taku
%			Added a check against jsps-abs-p1-header.
% 2017-09-02 Taku
%			Added sectionNo to the commands to make them compatible with 
%			\KLBeginSubjectWithHeaderCommands.
%			Use \KLJInt.
% 2018-09-01 Taku
%			Adjusted the heights of the headers by inserting \vspace{-3pt} and \rule.
%
\newcommand{\headerfont}{\fontsize{11}{11}\selectfont}
% ===== Headers =====================================
\newcommand{\JSPSVeryFirstPageStyle}[5]{%
%	Defines the header for the very first page of the form.
%	Called from \KLShumokuFirstPageStyle in form04_***.
%--------------------------------
%	#1: page style name
%	#2: 様式
%	#3: 研究種目名
%	#4: 項目名
%	#5: sectionNo
%--------------------------------
	\fancypagestyle{JSPSVeryFirstPageStyle}{% The name is not taken from #1, because 
		\fancyhf{}
		\fancyhead[L]{\hspace{-37pt}\headerfont#2\ 研究計画調書(添付ファイル項目)\\
				\rule{0pt}{18pt}\\}
%				\rule{0pt}{0pt}\\}
		\fancyhead[R]{\headerfont\textbf{#3\ \KLJInt{\thepage}}\vspace{-5pt}\\
			\rule{0pt}{0pt}\\}
%		\fancyhead[R]{\headerfont\textbf{#3\ \KLJInt{\thepage}\\}}
	}
	\thispagestyle{JSPSVeryFirstPageStyle}
}

\newcommand{\JSPSFirstSubjectPageStyle}[5]{%
%	Defines the header for the first page for the subject.
%	Called from \KLShumokuFirstPageStyle in form04_***.
%--------------------------------
%	#1: page style name
%	#2: 様式
%	#3: 研究種目名
%	#4: 項目名
%	#5: sectionNo
%--------------------------------
	\fancypagestyle{JSPSFirstSubjectPageStyle}{%
		\fancyhf{}
		\fancyhead[R]{\headerfont\textbf{#3\ \KLJInt{\thepage}}\vspace{-5pt}\\
			\rule{0pt}{0pt}\\}
%		\fancyhead[R]{\headerfont\textbf{#3\ \KLJInt{\thepage}\\}}
	}
	\thispagestyle{JSPSFirstSubjectPageStyle}
}

\newcommand{\JSPSDefaultPageStyle}[5]{%
%	Defines the default header for the subject.
%	Called from \KLShumokuDefaultPageStyle in form04_***.
%--------------------------------
%	#1: page style name
%	#2: 様式
%	#3: 研究種目名
%	#4: 項目名
%	#5: sectionNo
%--------------------------------
	\fancypagestyle{JSPSDefaultPageStyle}{%
		\fancyhf{}
		\fancyhead[L]{\headerfont\textbf{【#4(つづき)\ 】}\vspace{-7pt}\\}
		\fancyhead[R]{\headerfont\textbf{#3\ \KLJInt{\thepage}}\vspace{-5pt}\\
			\rule{0pt}{0pt}\\}
%		\fancyhead[R]{\headerfont\textbf{#3\ \KLJInt{\thepage}\\}}	
        }
        \pagestyle{JSPSDefaultPageStyle}
}

 % pieces
% form04_tokusui-2_header.tex
%=================================================================

% ===== Global definitions for the Kakenhi form ======================
% 基本情報
\newcommand{\様式}{様式S−1(2)}
\newcommand{\研究種目}{特別推進研究}
\newcommand{\研究種目後半}{ 新規(日本語)}
\ifthenelse{\isundefined{\研究種別}}{%
	\newcommand{\研究種別}{}
}{}%
\newcommand{\研究種目header}{\研究種目}

\newcommand{\KLMainFile}{jpa\_tokusui-2.tex}
\newcommand{\KLYoshiki}{tokusui-2_header}

% ===== Headers =====================================
\fancypagestyle{tokusui-2-1-p1-header}{%
	\fancyhf{}
	\fancyhead[L]{\hspace{-37pt}\headerfont{Form S-1 (2): Research Proposal Document (forms to be uploaded)}\\
				\rule{0pt}{0pt}\\}
	\fancyhead[R]{\headerfont{\textbf{Specially Promoted Research 2-1-( \thepage\ )}\\}}
}

\fancypagestyle{tokusui-2-1-header}{%
	\fancyhf{}
	\fancyhead[R]{\headerfont{\textbf{Specially Promoted Research 2-1-( \thepage\ )}\\}}
}

\fancypagestyle{tokusui-2-2-header}{%
	\fancyhf{}
	\fancyhead[R]{\headerfont{\textbf{Specially Promoted Research 2-2-( \thepage\ )}\\}}
}

\fancypagestyle{tokusui-2-3-header}{%
	\fancyhf{}
	\fancyhead[R]{\headerfont{\textbf{Specially Promoted Research 2-3-( \thepage\ )}\\}}
}

\fancypagestyle{tokusui-2-4-header}{%
	\fancyhf{}
	\fancyhead[R]{\headerfont{\textbf{Specially Promoted Research 2-4-( \thepage\ )}\\}}
}

%==========================================================
 % pieces
% ===== Global definitions for the Kakenhi form ======================
% 基本情報
%
%------ 研究課題名  -------------------------------------------
\newcommand{\研究課題名}{象の卵}

%----- 研究機関名と研究代表者の氏名-----------------------
\newcommand{\研究機関名}{逢坂大学}
\newcommand{\研究代表者氏名}{湯川秀樹}
\newcommand{\me}{\underline{\underline{H.~Yukawa}}} 
%---- 研究期間の最終年度 ----------------
\newcommand{\研究期間の最終元号年度}{8}  %令和で、半角数字のみ
%========================================

% inst_general.tex
%--------------------------------------------------------------------
% For writing instructions
%--------------------------------------------------------------------
\newcommand{\KLInstWOGeneral}[1]{%
	\noindent
 ーー Note ーーーーーーーーーーーーーーーーーーーーーーーーーーーーーー\\
		#1\\
 ーーーーーーーーーーーーーーーーーーーーーーーーーーーーーーーーーーーーーーーー
}

\newcommand{\KLInst}[1]{%
	\noindent
	\ifthenelse{\equal{#1}{}}{%
 ーー Note ーーーーーーーーーーーーーーーーーーーーーーーーーーーーーーー\\
	}{%
 ーー Note\textcircled{1} ーーーーーーーーーーーーーーーーーーーーーーーーーーーーーーー\\
		#1\\
		
	\noindent
 ーー Note\textcircled{2} ーーーーーーーーーーーーーーーーーーーーーーーーーーーーーーー\\
	}
}

\newcommand{\KLInstLine}[1]{%
	\ifthenelse{\equal{#1}{}}{%
 ーー Note ーーーーーーーーーーーーーーーーーーーーーーーーーーーーーー\\
	}{%
 ーー Note\textcircled{#1} ーーーーーーーーーーーーーーーーーーーーーーーーーーーーーーー\\
	}
}

\newcommand{\KLInstructionTitle}{%
	\textbf{\large Matters to be noted when preparing the Research Proposal Document}\\
}

\newcommand{\DeleteInstructions}[1]{%
	\textcolor{red}{Read the following important notes carefully before preparing this form. Delete this entire text box when filling in this form.\\
 (Delete \texttt{\textbackslash #1} and other text)}%
}

% local variables for \GeneralInstructions ------------------------
\newcommand{\留意事項内種目名}{}			% #1
\newcommand{\留意事項内記入要領名}{}		% #2
\newcommand{\留意事項内種目名削除用}{}	% #3

\newcommand{\KakenhiInstructions}{%
    \textbf{1.以下の内容を熟読・理解の上、研究計画調書を作成すること。\\
  ーーーーーーーーーーーーーーーーーーーーーーーーーーーーーーーーーーーーーーーーーーー\\
    \begin{small}
     科研費は、研究者の自由な発想に基づく全ての分野にわたる研究を格段に発展させることを目的とし、\\
    豊かな社会発展の基盤となる独創的・先駆的な研究を支援します。\\
     科研費では、応募者が自ら自由に課題設定を行うため、提案課題の学術的意義に加え、独自性や創造性\\
    が重要な評価ポイントになります。このため、「基盤研究」及び「若手研究」の研究計画調書様式では、\\
    学術の潮流や新たな展開などどのような「学術的背景」の下でどのような「学術的『問い』」を設定した\\
    か、当該課題の「学術的独自性や創造性」、「着想に至った経緯」、「国内外の研究動向と本研究の位置\\
    付け」はどのようなものか、などの記述を求めています。\\
     審査においては、総合審査又は二段階書面審査における審査委員間の議論・意見交換等により研究課題\\
    の核心を掴み、学術的な意義や独自性、創造性など学術的重要性を評価するとともに、実行可能性並びに\\
    研究遂行能力も含めて総合的に判断します。\\
     科研費に応募するに当たっては、上記に留意の上、公募要領や審査基準、様式の説明書き等を十分に確\\
    認し、審査委員に学術的重要性等が適切に伝わるように研究計画調書を作成してください。\\
    \end{small}%
  ーーーーーーーーーーーーーーーーーーーーーーーーーーーーーーーーーーーーーーーーーーー\\
    }
}

\newcommand{\GeneralInstructions}[3]{%
	\ifthenelse{\equal{#1}{}}{%
		\renewcommand{\留意事項内種目名}{研究計画調書}%
	}{%
		\renewcommand{\留意事項内種目名}{#1}%
	}%
	\ifthenelse{\equal{#2}{}}{%
		\renewcommand{\留意事項内記入要領名}{作成・記入要領}%
	}{%
		\renewcommand{\留意事項内記入要領名}{#2}%
	}%
	\ifthenelse{\equal{#3}{}}{%
		\renewcommand{\留意事項内種目名削除用}{\留意事項内種目名}%
	}{%
		\renewcommand{\留意事項内種目名削除用}{#3}%
	}%
	1. Read carefully the ``Procedures for Preparing and Entering a Research Proposal Document'' when preparing the document.\\
	2. The document should be written with font size 10-point or larger.\\
	3. The title and instructions on the upper part of each page should be left intact.\\
	4. Do not exceed the maximum number of pages specified in the instructions. In case blank page(s) occur, leave them as they are (do not eliminate any page).\\ 
 \DeleteInstructions{JSPSInstructions}\\
 ーーーーーーーーーーーーーーーーーーーーーーーーーーーーーーーーーーーーーーーー
}


\newcommand{\PapersInstructions}{%
 ーー Note ーーーーーーーーーーーーーーーーーーーーーーーーーーーーーーー\\
\begin{small}%
 1. 研究業績(論文、著書、産業財産権、招待講演等)は、網羅的に記載するのではなく、本研\\
   究計画の実行可能性を説明する上で、その根拠となる文献等の主要なものを適宜記載するこ\\
   と。\\
 2.研究業績の記述に当たっては、当該研究業績を同定するに十分な情報を記載すること。\\
   例として、学術論文の場合は論文名、著者名、掲載誌名、巻号や頁等、発表年(西暦)、著\\
   書の場合はその書誌情報、など。\\
 3.論文は、既に掲載されているもの又は掲載が確定しているものに限って記載すること。\\
 \DeleteInstructions{PapersInstructions}\\
\end{small}
 ーーーーーーーーーーーーーーーーーーーーーーーーーーーーーーーーーーーーーーーー\\
} % pieces
%inst_kiban_tokusui-2.tex
\newcommand{\JSPSInstructions}{%
	\\
	\KLInstLine{}
	\KLInstructionTitle
%	\DeleteInstructions{JSPSInstructions}\\
	\KLInstLine{1}
  1. 特別推進研究は人文社会・理工・生物の「系」の区分により、広い分野の委員構成\\
    で多角的視点から審査が行われることに留意の上、研究計画調書を作成すること。\\
	\KLInstLine{2}
	\GeneralInstructions{}{}{}
}

\renewcommand{\PapersInstructions}{%
	\KLInstLine{}
\begin{small}%
 1.研究代表者(研究分担者がいる場合は研究分担者も同様)の研究発表論文や著書、講演等の\\
   研究業績については、本欄ではなく「RECENT RESEARCH ACTIVITIES I (Publications)」\\
   及び「RECENT RESEARCH ACTIVITIES II (Invited Lectures and Talks, Prizes, etc.)」に\\
   主要なものを記載すること。\\
 2.本欄において、これまでの研究活動で得られた成果を示すに当たり、特定の具体的な研究業績\\
   (論文、著書、産業財産権、招待講演等)を明示する特段の必要がある場合は、当該業績を\\
   同定するに十分な情報を記載すること。\\
   例として、学術論文の場合は論文名、著者名、掲載誌名、巻号や頁等、発表年(西暦)、著書の場合は\\
   その書誌情報、など。\\
 3.論文は、既に掲載されているもの又は掲載が確定しているものに限って記載すること。\\
 \DeleteInstructions{PapersInstructions}\\
\end{small}
 ーーーーーーーーーーーーーーーーーーーーーーーーーーーーーーーーーーーーーーーー\\
} % pieces
% user07_header
% ===== my favorite packages ====================================
% ここに、自分の使いたいパッケージを宣言して下さい。
\usepackage{wrapfig}
%\usepackage{amssymb}
%\usepackage{mb}
%\DeclareGraphicsRule{.tif}{png}{.png}{`convert #1 `dirname #1`/`basename #1 .tif`.png}
\usepackage{lineno}

% ===== my personal definitions ==================================
% ここに、自分のよく使う記号などを定義して下さい。
\newcommand{\klpionn}{K_L \to \pi^0 \nu \overline{\nu}}
\newcommand{\kppipnn}{K^+ \to \pi^+ \nu \overline{\nu}}

% ----- 業績リスト用 -------------
\newcommand{\paper}[6]{%
	% paper{title}{authors}{journal}{vol}{pages}{year}
	\item ``#1'', #2, #3 {\bf #4}, #5 (#6).			% お好みに合わせて変えてください。
}

\newcommand{\etal}{\textit{et al.\ }}
\newcommand{\ca}[1]{*#1}	% corresponding author;   \ca{\yukawa}  みたいにして使う
\newcommand{\invitedtalk}{招待講演}

\newcommand{\yukawa}{H.~Yukawa}					% no underline
%\newcommand{\yukawa}{\underline{\underline{H.~Yukawa}}}	% with 2 underlines
\newcommand{\tomonaga}{S.~Tomonaga}

\newcommand{\prl}{Phys.\ Rev.\ Lett.\ }		% よく使う雑誌も定義すると楽

% ===== 欄外メモ ==================
\newcommand{\memo}[1]{\marginpar{#1}}
%\renewcommand{\memo}[1]{}	% 全てのメモを表示させないようにするには、行頭の"%"を消す


%\input{../../sample/simple/contents}	% skip
\input{pieces/hook5} % pieces

\begin{document}
\input{pieces/hook7} % pieces
%#Split: 01_purpose_background  
%#PieceName: p01_purpose_background
% p01_purpose_background_00.tex
\KLBeginSubject{01}{1}{研究目的、背景など}{7}{V}{1}{tokusui-2-1-p1-header}{tokusui-2-1-header}

\section{研究目的、背景など}
%    <<最大 7ページ>>

%s02_purpose_baclground_tokusui-2.tex
\noindent
\mbox{\textbf{(概要)}}\\
%		<<10行程度>>
%begin 研究目的や背景などの概要 ====================
	象の卵の研究の目的は...	
	象の卵の研究計画と方法は...
%end 研究目的や背景などの概要 ====================

\noindent
\rule{\linewidth}{1pt}\\
\noindent
\mbox{\textbf{(本文)}}
%begin 研究目的や背景など ====================
\JSPSInstructions	% <-- 留意事項。これは消すか、コメントアウトしてください。

\textbf{\\     *** 以下は、あくまで例です。真似しないでください。 ***\\
     *** 本文はもちろん、節の切り方や論理の組み方は   ***\\
     *** ご自分の気に入ったスタイルで書いてください。  ***}

	 象の卵の研究目的は...

	唯一無二。

%blahblah
An elephant was born from a big egg.
That big egg was hatched by its mother elephant.
That mother elephant was born from another egg.
That big egg was hatched by its mother elephant.
That mother elephant was born from another egg.
That big egg was hatched by its mother elephant.
That mother elephant was born from another egg.
That big egg was hatched by its mother elephant.
That mother elephant was born from another egg.
That big egg was hatched by its mother elephant.
That mother elephant was born from another egg.
That big egg was hatched by its mother elephant.
That mother elephant was born from another egg.
That big egg was hatched by its mother elephant.
That mother elephant was born from another egg.
That big egg was hatched by its mother elephant.
That mother elephant was born from another egg.
That big egg was hatched by its mother elephant.
That mother elephant was born from another egg.
That big egg was hatched by its mother elephant.
That mother elephant was born from another egg.
That big egg was hatched by its mother elephant.
That mother elephant was born from another egg.
That big egg was hatched by its mother elephant.
That mother elephant was born from another egg.
That big egg was hatched by its mother elephant.
That mother elephant was born from another egg.
That big egg was hatched by its mother elephant.
That mother elephant was born from another egg.
That big egg was hatched by its mother elephant.
That mother elephant was born from another egg.
That big egg was hatched by its mother elephant.
That mother elephant was born from another egg.
That big egg was hatched by its mother elephant.
That mother elephant was born from another egg.
That big egg was hatched by its mother elephant.
That mother elephant was born from another egg.
That big egg was hatched by its mother elephant.
That mother elephant was born from another egg.
That big egg was hatched by its mother elephant.
That mother elephant was born from another egg.
That big egg was hatched by its mother elephant.
That mother elephant was born from another egg.
That big egg was hatched by its mother elephant.
That mother elephant was born from another egg.
That big egg was hatched by its mother elephant.
That mother elephant was born from another egg.
That big egg was hatched by its mother elephant.
That mother elephant was born from another egg.
That big egg was hatched by its mother elephant.
That mother elephant was born from another egg.
That big egg was hatched by its mother elephant.
That mother elephant was born from another egg.
That big egg was hatched by its mother elephant.
That mother elephant was born from another egg.
That big egg was hatched by its mother elephant.
That mother elephant was born from another egg.
That big egg was hatched by its mother elephant.
That mother elephant was born from another egg.
That big egg was hatched by its mother elephant.
That mother elephant was born from another egg.
That big egg was hatched by its mother elephant.
That mother elephant was born from another egg.
That big egg was hatched by its mother elephant.
That mother elephant was born from another egg.
That big egg was hatched by its mother elephant.
That mother elephant was born from another egg.
That big egg was hatched by its mother elephant.
That mother elephant was born from another egg.
That big egg was hatched by its mother elephant.

%blahblah
An elephant was born from a big egg.
That big egg was hatched by its mother elephant.
That mother elephant was born from another egg.
That big egg was hatched by its mother elephant.
That mother elephant was born from another egg.
That big egg was hatched by its mother elephant.
That mother elephant was born from another egg.
That big egg was hatched by its mother elephant.
That mother elephant was born from another egg.
That big egg was hatched by its mother elephant.
That mother elephant was born from another egg.
That big egg was hatched by its mother elephant.
That mother elephant was born from another egg.
That big egg was hatched by its mother elephant.
That mother elephant was born from another egg.
That big egg was hatched by its mother elephant.
That mother elephant was born from another egg.
That big egg was hatched by its mother elephant.
That mother elephant was born from another egg.
That big egg was hatched by its mother elephant.
That mother elephant was born from another egg.
That big egg was hatched by its mother elephant.
That mother elephant was born from another egg.
That big egg was hatched by its mother elephant.
That mother elephant was born from another egg.
That big egg was hatched by its mother elephant.
That mother elephant was born from another egg.
That big egg was hatched by its mother elephant.
That mother elephant was born from another egg.
That big egg was hatched by its mother elephant.
That mother elephant was born from another egg.
That big egg was hatched by its mother elephant.
That mother elephant was born from another egg.
That big egg was hatched by its mother elephant.
That mother elephant was born from another egg.
That big egg was hatched by its mother elephant.
That mother elephant was born from another egg.
That big egg was hatched by its mother elephant.
That mother elephant was born from another egg.
That big egg was hatched by its mother elephant.
That mother elephant was born from another egg.
That big egg was hatched by its mother elephant.
That mother elephant was born from another egg.
That big egg was hatched by its mother elephant.
That mother elephant was born from another egg.
That big egg was hatched by its mother elephant.
That mother elephant was born from another egg.
That big egg was hatched by its mother elephant.
That mother elephant was born from another egg.
That big egg was hatched by its mother elephant.
That mother elephant was born from another egg.
That big egg was hatched by its mother elephant.
That mother elephant was born from another egg.
That big egg was hatched by its mother elephant.
That mother elephant was born from another egg.
That big egg was hatched by its mother elephant.
That mother elephant was born from another egg.
That big egg was hatched by its mother elephant.
That mother elephant was born from another egg.
That big egg was hatched by its mother elephant.
That mother elephant was born from another egg.
That big egg was hatched by its mother elephant.
That mother elephant was born from another egg.
That big egg was hatched by its mother elephant.
That mother elephant was born from another egg.
That big egg was hatched by its mother elephant.
That mother elephant was born from another egg.
That big egg was hatched by its mother elephant.
That mother elephant was born from another egg.
That big egg was hatched by its mother elephant.


	\vspace{1cm}
	\begin{thebibliography}{99}
		\bibitem{teramura} 寺村輝夫、「ぼくは王様 - ぞうのたまごのたまごやき」.
	\end{thebibliography}
%end 研究目的や背景など ====================
%====================================

% p01_purpose_background_01.tex
\KLEndSubject{V}


%#Split: 02_importance  
%#PieceName: p02_importance
% p02_importance_00.tex
\KLBeginSubject{02}{2}{2 挑戦的研究としての意義}{1}{F}{}{jsps-subject-header}{jsps-default-header}

\section{特別推進研究としての意義}
%    <<最大 7ページ>>

%s04_importance.tex
%begin 期待される研究成果と学術上の意義とインパクト ====================
	象の卵を発見し、その構造を解明する。
	この発見により、哺乳類は卵を産まないという学術の世界の
	「常識の殻」を文字通り打ち破ることができる。
	また、他分野の研究の場においても、古くからの「常識の殻」を
	打ち砕くきっかけとなり、科学全体が大きく前進するきっかけとなる。
%end 期待される研究成果と学術上の意義とインパクト ====================

% p02_importance_01.tex
\KLEndSubject{V}


%#Split: 03_plan  
%#PieceName: p03_plan
% p03_plan_00.tex
\KLBeginSubject{03}{3}{研究計画・方法}{7}{V}{1}{}{tokusui-2-3-header}

\section{研究計画・方法}
%    <<最大 7ページ>>

%s08_plan_tokusui
%begin 研究計画と方法 ====================
	象の卵の研究計画は...

	準備はしようとしている。
	多分できると思う。

	研究代表者と研究分担者の役割分担は...

	%blahblah
An elephant was born from a big egg.
That big egg was hatched by its mother elephant.
That mother elephant was born from another egg.
That big egg was hatched by its mother elephant.
That mother elephant was born from another egg.
That big egg was hatched by its mother elephant.
That mother elephant was born from another egg.
That big egg was hatched by its mother elephant.
That mother elephant was born from another egg.
That big egg was hatched by its mother elephant.
That mother elephant was born from another egg.
That big egg was hatched by its mother elephant.
That mother elephant was born from another egg.
That big egg was hatched by its mother elephant.
That mother elephant was born from another egg.
That big egg was hatched by its mother elephant.
That mother elephant was born from another egg.
That big egg was hatched by its mother elephant.
That mother elephant was born from another egg.
That big egg was hatched by its mother elephant.
That mother elephant was born from another egg.
That big egg was hatched by its mother elephant.
That mother elephant was born from another egg.
That big egg was hatched by its mother elephant.
That mother elephant was born from another egg.
That big egg was hatched by its mother elephant.
That mother elephant was born from another egg.
That big egg was hatched by its mother elephant.
That mother elephant was born from another egg.
That big egg was hatched by its mother elephant.
That mother elephant was born from another egg.
That big egg was hatched by its mother elephant.
That mother elephant was born from another egg.
That big egg was hatched by its mother elephant.
That mother elephant was born from another egg.
That big egg was hatched by its mother elephant.
That mother elephant was born from another egg.
That big egg was hatched by its mother elephant.
That mother elephant was born from another egg.
That big egg was hatched by its mother elephant.
That mother elephant was born from another egg.
That big egg was hatched by its mother elephant.
That mother elephant was born from another egg.
That big egg was hatched by its mother elephant.
That mother elephant was born from another egg.
That big egg was hatched by its mother elephant.
That mother elephant was born from another egg.
That big egg was hatched by its mother elephant.
That mother elephant was born from another egg.
That big egg was hatched by its mother elephant.
That mother elephant was born from another egg.
That big egg was hatched by its mother elephant.
That mother elephant was born from another egg.
That big egg was hatched by its mother elephant.
That mother elephant was born from another egg.
That big egg was hatched by its mother elephant.
That mother elephant was born from another egg.
That big egg was hatched by its mother elephant.
That mother elephant was born from another egg.
That big egg was hatched by its mother elephant.
That mother elephant was born from another egg.
That big egg was hatched by its mother elephant.
That mother elephant was born from another egg.
That big egg was hatched by its mother elephant.
That mother elephant was born from another egg.
That big egg was hatched by its mother elephant.
That mother elephant was born from another egg.
That big egg was hatched by its mother elephant.
That mother elephant was born from another egg.
That big egg was hatched by its mother elephant.
  % << only for demonstration. Please delete it or comment it out.
	%blahblah
An elephant was born from a big egg.
That big egg was hatched by its mother elephant.
That mother elephant was born from another egg.
That big egg was hatched by its mother elephant.
That mother elephant was born from another egg.
That big egg was hatched by its mother elephant.
That mother elephant was born from another egg.
That big egg was hatched by its mother elephant.
That mother elephant was born from another egg.
That big egg was hatched by its mother elephant.
That mother elephant was born from another egg.
That big egg was hatched by its mother elephant.
That mother elephant was born from another egg.
That big egg was hatched by its mother elephant.
That mother elephant was born from another egg.
That big egg was hatched by its mother elephant.
That mother elephant was born from another egg.
That big egg was hatched by its mother elephant.
That mother elephant was born from another egg.
That big egg was hatched by its mother elephant.
That mother elephant was born from another egg.
That big egg was hatched by its mother elephant.
That mother elephant was born from another egg.
That big egg was hatched by its mother elephant.
That mother elephant was born from another egg.
That big egg was hatched by its mother elephant.
That mother elephant was born from another egg.
That big egg was hatched by its mother elephant.
That mother elephant was born from another egg.
That big egg was hatched by its mother elephant.
That mother elephant was born from another egg.
That big egg was hatched by its mother elephant.
That mother elephant was born from another egg.
That big egg was hatched by its mother elephant.
That mother elephant was born from another egg.
That big egg was hatched by its mother elephant.
That mother elephant was born from another egg.
That big egg was hatched by its mother elephant.
That mother elephant was born from another egg.
That big egg was hatched by its mother elephant.
That mother elephant was born from another egg.
That big egg was hatched by its mother elephant.
That mother elephant was born from another egg.
That big egg was hatched by its mother elephant.
That mother elephant was born from another egg.
That big egg was hatched by its mother elephant.
That mother elephant was born from another egg.
That big egg was hatched by its mother elephant.
That mother elephant was born from another egg.
That big egg was hatched by its mother elephant.
That mother elephant was born from another egg.
That big egg was hatched by its mother elephant.
That mother elephant was born from another egg.
That big egg was hatched by its mother elephant.
That mother elephant was born from another egg.
That big egg was hatched by its mother elephant.
That mother elephant was born from another egg.
That big egg was hatched by its mother elephant.
That mother elephant was born from another egg.
That big egg was hatched by its mother elephant.
That mother elephant was born from another egg.
That big egg was hatched by its mother elephant.
That mother elephant was born from another egg.
That big egg was hatched by its mother elephant.
That mother elephant was born from another egg.
That big egg was hatched by its mother elephant.
That mother elephant was born from another egg.
That big egg was hatched by its mother elephant.
That mother elephant was born from another egg.
That big egg was hatched by its mother elephant.
  % << only for demonstration. Please delete it or comment it out.
		
%end 研究計画と方法 ====================

% p03_plan_01.tex
\KLEndSubject{V}


%#Split: 04_competence  
%#PieceName: p04_competence
% p04_competence_00.tex
\KLBeginSubject{04}{4}{応募者の研究遂行能力と研究環境}{7}{V}{1}{}{tokusui-2-4-header}

\section{応募者の研究遂行能力と研究環境}
%    <<最大 7ページ>>

% s13_competence_tokusui
%begin 特推研究代表者の研究遂行能力と研究実績 ====================
	応募者は過去20年間、7つの海を隅から隅まで航海し、
	浅瀬から深海まで潜り、文字通り東西南北上下の3次元で
	シロナガスクジラの卵の探索を行ってきた(業績\ref{pub:whale})。
	シロナガスクジラに飲み込まれそうになったり、海賊に捕まるなどの危険な目にも
	あったが、それにもめげず、研究を遂行してきた強靭な能力を有する。

\PapersInstructions	% <-- 研究業績留意事項。これは消すか、コメントアウトしてください。
%end 特推研究代表者の研究遂行能力と研究実績 ====================

% p04_competence_01.tex
\KLEndSubject{V}


%#Split: 99_tail
\input{pieces/hook9} % pieces
\end{document}

