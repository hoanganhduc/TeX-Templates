%==================================================
% kakenhi7.sty
%==================================================
% v1
% Minimum amount of macros for writing Kakenhi forms.
%
% 2005-10-24 Taku Yamanaka, Physics Dept. Osaka Univ.
%		taku@hep.sci.osaka-u.ac.jp
% 		Macros such as XYBC, etc. were imported from Kakenhi Macro at
% 		http://www.yukawa.kyoto-u.ac.jp/contents/researcher/kakenhi.html .
% 2006-06-04 Taku
%		Added macros to draw boxes if \DrawBox is in the source.
%		This is useful when designing the LaTeX forms.
% 2006-06-14 Taku
%		Added LaTeX macros to add costs 
%		(\KLResetGrandSum, \KLCostItem, \KLSum, \KLGrandSum).
%		Added a macro \Number to supply commas every 3 digits (imported from kkh.mac).
% 2006-06-25 Taku
%		Added \KLTabC, \KLTabR, \KLTabRNum to specify alignments in tables.
%		Please note that \phantom{...} is required for the column, 
%		or otherwise somehow p{...mm} is ignored.
% 2006-08-13 Taku
%		Added \KLItemNumUnitCost . This requires calc.sty.
% 2006-09-06 Taku
%		Added KLGrandTotalValue to add ALL the costs.
%		Also added \KLPrintGrandTotal to print the total on the console.
% 2006-09-09 Taku
%		Added macros to handle "efforts".
% 2006-09-10 Taku
%		Added \KLnewcounter to make series of counters.
%		Modified \KLResetGrandSum and \KLSum to add the sum for 
%		each category and year.
% 2006-09-11 Taku
%		Added \NumC to display LaTeX counter with commas.
% 2006-09-12 Taku
%		Added simple macros to make group table.
% 2006-09-23 Taku
%		Added the 'Fair License' notification.
% 2006-10-22 Taku
%		Initialize KLNumPeople to -1, so that the first header row will not be included in the count.
%====================================================
% v2
% 2007-03-24 Taku
%		Instead of using Kakenhi Macros to position items, 
%		switched to a new method using
%		"picture" environment.  The origin of the coordinate is set to 
%		the lower left corner of the paper.  The positions are given in "points", 
%		as can be read by gv. These methods were suggested by 
%		Tsutomu Sakurai at Saitama Univ..
% 2007-03-30 Taku
%		The sum of each category and year is made using 
%		macros \KLItemCost, etc., instead of \KLSum.  This is a step toward
%		automatically aligning category sums in the same year, in some forms.
% 2007-04-02 Taku
%		Added \KLBudgetMiniTabular, \KLMiniSum, etc. to handle
%		budget tables with multiple category columns.
% 2007-05-04 Taku
%		Added \multicolumnDottedLine .
%==================================================
% v3 
% 2007-08-14 Taku
%		Simplified page handling, by introducing \KLBeginSinglePage,
%		\KLPageRange, etc..
% 2007-09-01 Taku
%		Added a new command, \KLItemNumUnitCostLocation , 
%		and \KLAddCost to clean things up.
% 2007-09-06 Taku
%		Set \KLEven/OddLeft/RightEdge parameters in \KLWaterMark.
%		Without it, if \KLLeftEdge or \KLRightEdge is used inside watermark,
%		it generated a very obscure error message, which was hard to track down.
% 2007-09-09 Taku
%		Removed clearring \thiswatermark in \KLClearWaterMarks.
% 2007-09-12 Taku
%		Added \KLPriorityItemNumUnitCostTwo for Tokusui.
% 2007-09-14 Taku
%		Added \KLItemNumUnitCostTwo for tokutei_koubo.
% 2007-09-17 Taku
%		In \KLAddCost, costs are added only if it is within the defined year range.
% 2008-09-02 Taku
%		  Added \KLMonthPriorityItemNumUnitCostTwo for tokutei_keizoku.
% 2008-09-07 Taku
%		   Use \KLJFY to print year in budget tables.
% 2008-10-21 Taku
%			Added KLItemNumUnitCostInParen for shorei.
% 2009-09-03 Taku
%			Added \dottedLine .
%==================================================
% v4
% 2009-09-06 Taku
%			Added macros for partial typesetting.
% 2009-09-12 Taku
%			Added macros for showing coordinates and edges.
% 2009-09-13 Taku
%			Added macros to show boxes and minipage frames and their corner coordinates.
% 2010-03-04 Taku
%			Moved macros for calculating lengths and positions from form07_header.tex to here.
% 2010-04-11 Taku
%			Added \KLItemCostOne for jisedai, and necessary flags to print budget sums before 
%			the detailed budget table.
%==================================================
% v5
% 2011-10-20 Taku
%			By using array package within tabular environment, the following macros were simplified:
%			\KLTabC, \KLTabR, \KLBudgetMiniTabular, 
%			New Macros:
%			\KLCC, \KLCR
% 2011-10-24 Taku
%			Removed using \KLTabR from most of the budget tables.
% 2011-10-26 Taku
%			Added KLMyBudget.
%			Modified KLYearItemNumUnitCostTwo to just put the JFY if the second item is blank.  (For Kiban S)
% 2012-03-10 Taku
%			Changed the tabcolsep for \KLMyBudget to 0pt.
% 2012-08-14 Taku
%			Moved xxx_forms_pdf and _eps directories to under mother.
% 2012-09-09	Taku
%			Added \KLbibitem(B) and \KLcite(B).
% 2012-09-16	Taku
%			Added \KLOtherApplication, \KLOtherApplicationReasons, and \KLOtherFundReasons
%			for tokusui (and maybe for others in the future).
%==================================================
% v6
% 2012-11-10	Taku
% 			Modified \KLOtherApplicationReasons and \KLOtherFundReasons to make the tables compact.
%			These were in hook3.tex for the 2013 version.
%			Added \KLOtherApplicationDiff for many shumokus.
%			Removed \KLbibitemB and \KLciteB.
%			Changed \KLbibitem to use a dedicated column for numbering.
% 2012-11-11	Taku
%			Added \KLItemSetCostLocationInfo and \KLItemCostInfo.
% 2013-09-19	Taku
%			Added \KLOtherPD and \KLOtherPDShort to enter JSPS PD for other funds.
% 2013-10-02	Taku
%			Changed \KLbibItem, not to use a dedicated column for numbering, 
%			because otherwise the label defined in \label{...} cannot be used in @currentlabel.
% 2014-09-22	Taku
%			Added \KLCL for filling narrow tabular cells in English.  
%			(Suggested by Frank Bennett.)
% 2014-11-08	Taku
%			Added an instruction in \KLCheckPageLimit.
% 2015-08-23	Taku
%			Added NumCk to show numbers divided by 1000 (truncated).
% 2015-08-24	Taku
%			Introduced \KLLongPage and \KLSimpleLongPage to offer floating environment in 
%			single-page-frames.
%=====================================================
% v7
%	Frames are gone!  Simplify kakenhiLaTeX to benefit from 
%	the new style.
% 2017-05-03	Taku
%			Added \KLAnotherFund .
% 2017-05-27	Taku
%			Made \KLBeginSubject and \KLEndSubject to handle new 
%			mother file style.
% 2017-08-17 Taku
%	Added \KLItemCostNoYear and \KLEndBudgetNoYear for kokusai_kyoudou.
% 2017-08-19	Taku
%			Updated \KLBeginSubject and \KLEndSubject to handle 
%			various headers.
% 2017-08-20	Taku
%			\KLBeginSubject calls \KLFirstPageStyle and \KLDefaultPageStyle
%			which should be defined for each shumoku (or JSPS/MEXT).
%			This is to pass the subject name etc. to the header.
% 2017-08-27	Taku
%			Set section number in \KLBeginSubject.
% 2017-08-29	Taku
%			Moved over KLShumokuFirstPageStyle and KLShumokuDefaultPageStyle.
%			Added jsps-abs-p1-header, jps-abs-subject-header, and 
%			jsps-abs-default-header as arguments to \KLShumoku***Header.
% 2017-09-02	Taku
%			Added \KLBeginSubjectWithHeaderCommands for more flexible header style.
% 2017-09-03	Taku
%			Added \vspace*{-4mm} after \includegraphics in \KLBeginSubject*.
% 2017-09-05	Taku
%			Removed now-the-old commands.
% 2020-01-02	Taku
%			Added more numbers to \KLJint for gakuhen_a.
% 2020-01-15	Taku
%			Changed the horizontal position (none --> -3mm) 
%			and the width of the top boxes for headers (\linewidth --> 1.02\linewidth)
%			to reproduce the original boxes.  
%			This became necessary because the margins were set correctly in form02_header.tex.
% 2021-01-29	Taku
%			Set section counter and reset subsection counter only if the section # is given for
%			\KLBeginSubject and \KLBeginSubjectWithHeaderCommands .
%=====================================================

%=====================================================
% Macro to supply commas every 3 digits (up to 9 digits)
%	Imported from kkh.mac for Kakenhi Macro.
%=====================================================
%
\newif\ifNumWithCommas \NumWithCommastrue
\def\NumWithCommas{\NumWithCommastrue}
\def\NumWithoutCommas{\NumWithCommasfalse}
\newcount\Numa
\newcount\Numb
\def\Numempty{}%output blank if "-0" is given
\def\Number#1{\edef\Numpar{#1}\ifx\Numempty\Numpar\else%
\ifNumWithCommas\Numa=#1\relax
\ifnum\Numa>999999\divide\Numa by 1000000
\number\Numa,%
\multiply\Numa by -1000000\advance\Numa by #1\relax
\Numb=\Numa\divide\Numa by 1000
\ifnum\Numa<100 \ifnum\Numa<10 0\fi0\fi\number\Numa,%
\multiply\Numa by -1000\advance\Numa by \Numb
\ifnum\Numa<100 \ifnum\Numa<10 0\fi0\fi\number\Numa%
\else\ifnum\Numa>999\divide\Numa by 1000
\number\Numa,%
\multiply\Numa by -1000\advance\Numa by #1\relax
\ifnum\Numa<100 \ifnum\Numa<10 0\fi0\fi\number\Numa%
\else\number\Numa\fi\fi\else\number#1\fi\fi}

%======================================================
% Macro to display LaTeX counter with commas every 3 digits.
%======================================================
\newcommand{\NumC}[1]{\Number{\value{#1}}}

\newcounter{kyen}
\newcommand{\NumCk}[1]{%
	\setcounter{kyen}{\arabic{#1}/1000}
	\Number{\value{kyen}}
}

%======================================================
% Macros to align (right-justify, center) elements within a tabular cell
% whose width is defined by p{...}.
% 2006-06-25 Taku
% 	These are necessary, because the cell width should be given explicitly
% 	by p{...mm} to match the given table in a tabular environment.  
% 	One could allocate a width with \phantom{...},
% 	but it is a little tricky, since it depends on the font size.
%======================================================

%---------------------------------------------------------------------
% Center text within a tabular cell allocated by p{...}
%\newcommand{\KLTabC}[1]{\multicolumn{1}{c}{#1}}
\newcommand{\KLTabC}[1]{\centering\arraybackslash#1}
% This new method does not require a dummy table row to put them in correct columns.
%
% This should be used in tabular definition, as:
%	\begin{tabular}[t]{>{\KLCC}p{30pt}p{50pt}}
\newcommand{\KLCC}{\centering\arraybackslash}

%---------------------------------------------------------------------
% Right justify text within a tabular cell allocated  by p{...}
%\newcommand{\KLTabR}[1]{\multicolumn{1}{r@{\ }}{#1}}
\newcommand{\KLTabR}[1]{\raggedleft\arraybackslash#1}

% This should be used in tabular definition, as:
%	\begin{tabular}[t]{>{\KLCR}p{30pt}p{50pt}}
\newcommand{\KLCR}{\raggedleft\arraybackslash}%

%---------------------------------------------------------------------
% Right justify number (with comma every 3 digits) 
% within a tabular cell allocated by p{...}
\newcommand{\KLTabRNum}[1]{\KLTabR{\Number{#1}}}

%---------------------------------------------------------------------
% Left justify text within a tabular cell allocated by p{...}
% This should be used in tabular definition, as:
%	\begin{tabular}[t]{>{\KLCL}p{30pt}p{50pt}}
\newcommand{\KLCL}{\raggedright\arraybackslash}%

%=================================================
%  counter tools
%=================================================
\newcounter{KLtmp}

%------------------------------------------------------------------------------
% Makes a set of counters, with prefix #1, followed by 
% suffix ranging from 0 to #2 - 1.
% For example, \KLnewcounter{mine}{3} makes counters
% mine0, mine1, and mine2 .
%-------------------------------------------------------------------------------
\newcommand{\KLnewcounter}[2]{
	\setcounter{KLtmp}{0}
	
	\whiledo{\value{KLtmp} < #2}{
		\newcounter{#1\arabic{KLtmp}}
		\stepcounter{KLtmp}
	}
}

%------------------------------------------------------------------------------
% Dumps the contents of the counters.
%------------------------------------------------------------------------------
\newcommand{\KLdumpcounter}[2]{
	\setcounter{KLtmp}{0}
	
	\whiledo{\value{KLtmp} < #2}{
		#1\arabic{KLtmp} : \arabic{#1\arabic{KLtmp}}\\
		\stepcounter{KLtmp}
	}
}

%=======================================================
% LaTeX macros to add costs.
%	2006-06-14 Taku Yamanaka
%=======================================================
\newcounter{KLCost}				% to calculate cost = #units x unit cost
\newcounter{KLGrandTotalValue}		% for the grand total of all the categories in all years
\setcounter{KLGrandTotalValue}{0}

\newcommand{\KLCostCategory}{KLequipments}
\newcounter{KLYearCount}
\newcounter{KLPrintYear}

% Make counters for annual sums for each category-----------------------
\newcommand{\KLMaxYear}{8}
\KLnewcounter{KLequipments}{\KLMaxYear}
\KLnewcounter{KLexpendables}{\KLMaxYear}
\KLnewcounter{KLdomestic}{\KLMaxYear}
\KLnewcounter{KLforeign}{\KLMaxYear}
\KLnewcounter{KLtravel}{\KLMaxYear}
\KLnewcounter{KLgratitude}{\KLMaxYear}
\KLnewcounter{KLmisc}{\KLMaxYear}
\KLnewcounter{KLAnnualSum}{\KLMaxYear}

%------------------------------------------------
% Add up the given cost to category-year sum, category sum, year-sum, and total.
% 2007-09-01 Taku
% 2007-09-17 Taku: Add costs only if it is within the defined year range.
%------------------------------------------------
\newcommand{\KLAddCost}[1]{%
	\ifthenelse{\value{KLYearCount} > \value{KLMaxYearCount}}{%
		%pass
	}{%
		\addtocounter{\KLCostCategory0}{#1}%
		\addtocounter{\KLCostCategory\arabic{KLYearCount}}{#1}%
		\addtocounter{KLAnnualSum\arabic{KLYearCount}}{#1}%
		\addtocounter{KLAnnualSum0}{#1}%
		\ifthenelse{\equal{\KLCostCategory}{KLdomestic}}{%
			\addtocounter{KLtravel0}{#1}%
			\addtocounter{KLtravel\arabic{KLYearCount}}{#1}%
		}{}%
		\ifthenelse{\equal{\KLCostCategory}{KLforeign}}{%
			\addtocounter{KLtravel0}{#1}%
			\addtocounter{KLtravel\arabic{KLYearCount}}{#1}%
		}{}%
	}%
}


\newcommand{\KLClearWaterMarks}{%
	%--empty watermarks
	\watermark{}
%	\thiswatermark{}
	\rightwatermark{}
	\leftwatermark{}
}

\newcommand{\KLInput}[1]{%	The macros defined inside the file are only valid within the file.
	\begingroup
	\input{#1}
	\endgroup
}

%================================
% For 2017 new style without frames
%================================
\newcommand{\KLShumokuFirstPageStyle}[5]{%
%	Defines the header for the first page.
%	Called from \KLBeginSubject.
%--------------------------------
%	#1: page style name
%	#2: 様式
%	#3: 研究種目名
%	#4: 項目名
%	#5: sectionNo
%--------------------------------
	\ifthenelse{\equal{#1}{jsps-p1-header}}{%
		\JSPSVeryFirstPageStyle{#1}{#2}{#3}{#4}{#5}
	}{%
		\ifthenelse{\equal{#1}{jsps-abs-p1-header}}{%
			\JSPSVeryFirstPageStyle{#1}{#2}{#3 概要}{#4}{#5}
		}{%
            		\ifthenelse{\equal{#1}{jsps-subject-header}}{%
            			\JSPSFirstSubjectPageStyle{#1}{#2}{#3}{#4}{#5}
            		}{%
				\ifthenelse{\equal{#1}{jsps-abs-subject-header}}{%
            				\JSPSFirstSubjectPageStyle{#1}{#2}{#3 概要}{#4}{#5}
				}{%
                    			\thispagestyle{#1}
				}
            		}
		}
	}
}

\newcommand{\KLShumokuDefaultPageStyle}[5]{%
%	Defines the default header.
%	Called from \KLBeginSubject.
%--------------------------------
%	#1: page style name
%	#2: 様式
%	#3: 研究種目名
%	#4: 項目名
%	#5: sectionNo
%--------------------------------
	\ifthenelse{\equal{#1}{jsps-default-header}}{%
		\JSPSDefaultPageStyle{#1}{#2}{#3}{#4}{#5}
	}{%
		\ifthenelse{\equal{#1}{jsps-abs-default-header}}{%
			\JSPSDefaultPageStyle{#1}{#2}{#3 概要}{#4}{#5}
		}{%
            		\pagestyle{#1}
		}
	}
}

\newcommand{\KLSubjectName}{}
\newcommand{\KLSubjectMaxPages}{}
\newcommand{\KLSubjectEndPage}{}
\newcounter{KLSubjectEndPage}
\setcounter{KLSubjectEndPage}{0}

\newcommand{\KLSubjectCheckNPages}{%
%	\arabic{page}, \arabic{KLSubjectEndPage}\\
	\ifthenelse{\value{page}>\value{KLSubjectEndPage}}{
		{\LARGE「\KLSubjectName」は \KLSubjectMaxPages\ ページ以内で書いてください。}
		\clearpage
	}{%
	}
}

\newcommand{\KLSubjectAdvancePages}{%
	\renewcommand{\KLSubjectEndPage}{\value{KLSubjectEndPage}}
	\ifthenelse{\value{page}<\KLSubjectEndPage}{%
		\phantom{x}\clearpage
	}{}
	% Advance page if necessary
	\ifthenelse{\value{page}<\KLSubjectEndPage}{%
		\phantom{x}\clearpage
	}{}
	% Advance page if necessary
	\ifthenelse{\value{page}<\KLSubjectEndPage}{%
		\phantom{x}\clearpage
	}{}
	% Advance page if necessary
	\ifthenelse{\value{page}<\KLSubjectEndPage}{%
		\phantom{x}\clearpage
	}{}
	% Advance page if necessary
	\ifthenelse{\value{page}<\KLSubjectEndPage}{%
		\phantom{x}\clearpage
	}{}
}	

\newcommand{\KLJInt}[1]{%
% Returns full-width numerical character.
	\ifthenelse{\equal{#1}{1}}{1}{%
	\ifthenelse{\equal{#1}{2}}{2}{%
	\ifthenelse{\equal{#1}{3}}{3}{%
	\ifthenelse{\equal{#1}{4}}{4}{%
	\ifthenelse{\equal{#1}{5}}{5}{%
	\ifthenelse{\equal{#1}{6}}{6}{%
	\ifthenelse{\equal{#1}{7}}{7}{%
	\ifthenelse{\equal{#1}{8}}{8}{%
	\ifthenelse{\equal{#1}{9}}{9}{%
	\ifthenelse{\equal{#1}{10}}{10}{%
	\ifthenelse{\equal{#1}{11}}{11}{%
	\ifthenelse{\equal{#1}{12}}{12}{%
	\ifthenelse{\equal{#1}{13}}{13}{%
	\ifthenelse{\equal{#1}{14}}{14}{%
	\ifthenelse{\equal{#1}{15}}{15}{%
	\ifthenelse{\equal{#1}{16}}{16}{%
	\ifthenelse{\equal{#1}{17}}{17}{%
	#1}}}}}}}}}}}}}}}}}%
}


\newcommand{\KLBeginSubject}[8]{%
%----------------------------------------------------
%	#1: subjectNo
%	#2: sectionNo
%	#3: sectionJ
%	#4: maxPages
%	#5: pageLengthStyle ('V' for variable, 'F' for fixed)
%	#6: pageCounter (set page counter to this value if the argument exists.
%	#7: subjectFirstPageHeader (header for the first page)
%	#8: defaultPageHeader
%----------------------------------------------------
	\ifthenelse{\equal{#2}{}}{%
	}{%
	    	\setcounter{section}{#2}
	    	\setcounter{subsection}{0}
	}
	\setcounter{subsubsection}{0}
	\renewcommand{\KLSubjectName}{#3}
	\renewcommand{\KLSubjectMaxPages}{#4}
	
	\ifthenelse{\equal{#6}{}}{%
	}{%
		\setcounter{page}{#6}
	}
	
	\setcounter{KLSubjectEndPage}{\value{page}}
	\addtocounter{KLSubjectEndPage}{#4}
	
	\ifthenelse{\equal{#7}{}}{%
		% pass
	}{%
		\KLShumokuFirstPageStyle{#7}{\様式}{\研究種目header}{#3}{#2}
	}
	
	\ifthenelse{\equal{#8}{}}{%
		% pass
	}{%
		\KLShumokuDefaultPageStyle{#8}{\様式}{\研究種目header}{#3}{#2}
	}
	
	\noindent
	\hspace{-3mm}
	\includegraphics[width=1.03\linewidth]{subject_headers/\KLYoshiki_#1.pdf}\\
	\vspace*{-4mm}
}

\newcommand{\KLNullHeader}[5]{}
% Dummy command for No header.
% This was introduced to avoid error caused in statement \ifthenelse{\equal{#8}{}} .

\newcommand{\KLBeginSubjectWithHeaderCommands}[8]{%
%----------------------------------------------------
%	#1: subjectNo
%	#2: sectionNo
%	#3: sectionJ
%	#4: maxPages
%	#5: pageLengthStyle ('V' for variable, 'F' for fixed)
%	#6: pageCounter (set page counter to this value if the argument exists.
%	#7: LaTeX command for subjectFirstPageHeader (header for the first page)
%	#8: LaTeX command for defaultPageHeader
%----------------------------------------------------
	\ifthenelse{\equal{#2}{}}{%
	}{%
		\setcounter{section}{#2}
		\setcounter{subsection}{0}
	}
	\setcounter{subsubsection}{0}
	\renewcommand{\KLSubjectName}{#3}
	\renewcommand{\KLSubjectMaxPages}{#4}
	
	\ifthenelse{\equal{#6}{}}{%
	}{%
		\setcounter{page}{#6}
	}
	
	\setcounter{KLSubjectEndPage}{\value{page}}
	\addtocounter{KLSubjectEndPage}{#4}
	
	#7{#7}{\様式}{\研究種目header}{#3}{#2}
	#8{#8}{\様式}{\研究種目header}{#3}{#2}
	
	\noindent
	\hspace{-3mm}
	\includegraphics[width=1.03\linewidth]{subject_headers/\KLYoshiki_#1.pdf}\\
	\vspace*{-4mm}
}

\newcommand{\KLEndSubject}[1]{%
%	#1: pageLengthStyle ('V' for variable, 'F' for fixed)
		\clearpage % This should be done to update page counter for checking.
		\KLSubjectCheckNPages
		\ifthenelse{\equal{#1}{F}}{%
			\KLSubjectAdvancePages
		}{%
		}
}

%==================================================
% Miscellaneous macros
%==================================================

%----------------------------------------------------------------------
% Draw dotted lines across a multiple column table
%----------------------------------------------------------------------
\newcommand{\multicolumnDottedLine}[1]{%
%	\multicolumn{#1}{@{\hspace{-2mm}}c}{\dotfill}\\%
	\multicolumn{#1}{@{}c}{\dotfill}\\%
}

\newcommand{\dottedLine}{%
	\\\noindent
	\dotfill\\
}

%----------------------------------------------------------------------
% Solid line
%----------------------------------------------------------------------
\newlength{\KLLineLength}
\newcommand{\solidLine}[1]{
%----------- keep an empty line between here and \noindent so that it works after normal text and list.

	\noindent
	\hspace*{-10pt}
	\rule[10pt]{\textwidth}{#1}% #1 = 0.5pt, ....
	\vspace*{-10pt}
}

\newcommand{\KLLine}{%
	\solidLine{1pt}
}

%----------------------------------------------------------------------
% publication list (Thanks to Tetsuo Iwakuma [bulletin board #876])
%----------------------------------------------------------------------
\newcounter{KLBibCounter}

\makeatletter	
	\newcommand{\KLbibitem}{%
		\stepcounter{KLBibCounter}%
		\let \@currentlabel \theKLBibCounter
		\arabic{KLBibCounter}. %
	}
\makeatother

\newcommand{\KLcite}[1]{[\ref{#1}]}

%==================================================
%Fair License

%<Copyright Information>

%Usage of the works is permitted provided that this
%instrument is retained with the works, so that any entity
%that uses the works is notified of this instrument.

%DISCLAIMER: THE WORKS ARE WITHOUT WARRANTY.

%[2004, Fair License: rhid.com/fair]
%==================================================
% You may edit/modify this package at your own risk.
% If there are important fixes or changes that you think should be 
% reflected in the standard distribution, please notify:
%	taku@hep.sci.osaka-u.ac.jp  .
%==================================================
