%#Split: 06_abilities_buntansha % extract 
%#PieceName: p06_abilities_buntansha
% p06_abilities_buntansha_00.tex
\KLBeginSubject{05}{5}{研究者調書(研究分担者)}{1}{F}{}{kiban-s-pub-subject-header}{kiban-s-pub-header}

\section{研究者調書(研究分担者)}
%    <<最大 1ページ>>

% s14_abilities_S_b
% s14_pub_S_commands_b
% begin 研究者情報 ==========================
\renewcommand{\研究者氏名}{朝永振一郎}
\renewcommand{\研究者氏名ふりがな}{ともなが しんいちろう}
\renewcommand{\研究者生年月日の年}{1900}
\renewcommand{\研究者生年月日の月}{4}
\renewcommand{\研究者生年月日の日}{31}
\renewcommand{\研究者年齢}{137}
\renewcommand{\研究者所属機関部局職}{\small{江戸文理大学\\ 理学部\\ 名誉教授}}	% use \tiny if necessary
\renewcommand{\研究者学位}{理学博士}
% end 研究者情報 ==========================

% s14_pub_s_picture
\renewcommand{\KLXi}{117}
\renewcommand{\KLYi}{-35}
\renewcommand{\KLYii}{-47}

\thiswatermark{%
	\begin{picture}(0,0)(0,0)
		\put(\KLXi, -32){\tiny{(\研究者氏名ふりがな})}
		\put(\KLXi, \KLYii){\研究者氏名}
		\put(343, \KLYi){\small{\研究者生年月日の年}}
		\put(383, \KLYi){\small{\研究者生年月日の月}}
		\put(401, \KLYi){\small{\研究者生年月日の日}}
		\put(405, \KLYii){\small{\研究者年齢}}
		\put(116, -61){%
			\parbox[t][50pt]{142pt}{\研究者所属機関部局職}
		}
		\put(330, -65){\研究者学位}
	\end{picture}
}
 % pieces

\vspace*{-1cm}
%begin 基盤S研究分担者の研究遂行能力及び研究環境 ====================
	\noindent
%	\textbf{(1)研究分担者のこれまでの研究活動}\\
	\textbf{(1) Co-I's hitherto research activities}\\
\PapersInstructions	% <-- 研究業績留意事項。これは消すか、コメントアウトしてください。

%	私は毎晩超新星を探索するために
%	目を皿のようにして望遠鏡をのぞき、
%	前夜の写真と新しい写真をそれぞれ別の目で見てきた。
%	そのために目の大きさと立体視能力では誰にも引けを取らない。
%\\

	\noindent
%	\textbf{(2)研究分担者の研究環境}\\
	\textbf{(2) Co-I's research environments}\\
%	天体観測用の広視野の望遠鏡はあるので、それを携えて地球周回軌道に行けば、
%	象の卵の探索はできる。
%	また、季節によらず約45分観測しては約45分寝られるので、
%	実質ほぼ24時間体制で探索を続けられる。
%	ただし、暗いと目が覚め、明るいと眠くなる長年の習慣を直す必要はある。

%end 基盤S研究分担者の研究遂行能力及び研究環境 ====================
%begin 研究業績リスト ====================
%	\begin{enumerate}
%		\paper{Supernova detection with human neural network}
%			{\me}{Astrophysics}{1234}{5678}{2019}
%			\label{pub:supernova}	% \labelをつければ、番号を引用できます。
%			
%		% 下のように書いてもいいけど、めんどくさいし、表示の仕方を変えようとしたら大変。
%		\item \label{pub:share} ``象の卵の抱卵の分担'',
%				\me, 
%				\underline{皇帝ペンギン} {\it et al.},
%				J.\ Eggs {\bf 123}, 456 (2018).
%					
%	\end{enumerate}
%end 研究業績リスト ====================

% p06_abilities_buntansha_01.tex
\KLEndSubject{F}


