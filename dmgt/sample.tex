\documentclass{dmgt}


%% do not change the next command
%% \setarticle{1}{xx}{yyyy}

\newauthor{%
John Rambo}{%
J. Rambo}{%
Affilation line 1\\
Affilation line 2\\
Affilation line 3}[%
Author1Email@domain.com]

\newauthor{%
John McClane}{%
J. McClane }{%
Affilation line 1\\
Affilation line 2\\
Affilation line 3}[%
Author2Email@domain.com]

\title{Sample article based on \texttt{\journalacronym{}} class}

\keywords{Type Keywords of your paper here}

\classnbr{Type \mscyear{} Mathematics Subject Classification of your paper here}


\begin{document}

\begin{abstract}
This sample article contains typical elements of article: definitions, theorems, proofs etc. 
\end{abstract}

\section{Introduction}

Here we have some definitions.

\begin{dnt}[\cite{gtwa}]
A graph is said to be \emph{embeddable in the plane} or \emph{planar}, if it can be drawn in the plane
so that its edges intersect only at their ends.
\end{dnt}


\begin{theorem}
$K_5$ is not planar.
\end{theorem}

\begin{proof}
See \cite{gtwa}.

\end{proof}

\begin{theorem}[(Eulers's formula)]
If $G$ is a connected plane graph, then 
$$v-e+f=2,$$
where $v$ -- number of vertices of $G$, $e$ -- number of edges of $G$ and $f$ -- number of faces of $G$.
\end{theorem}

\begin{proof}[Proof of Euler's formula]
See \cite{gtwa}.

\end{proof}

\begin{theorem}[(Kuratowski)]
A graph is planar if, and only if it contains no subdivision of $K_5$ or $K_{3,3}$.
\end{theorem}

\begin{proof}[Proof of Kuratowski's theorem]
In the proof we need two lemmatas:

\begin{lemma}\label{lemma1}
Lemma 1.
\end{lemma}

\begin{proof}[Proof of lemma {\bf \ref{lemma1}}]
Proof inside other proof is ended with white square.

\end{proof}


\begin{lemma}\label{lemma2}
Lemma 2.

\end{lemma}

\begin{proof}[Proof of lemma {\bf \ref{lemma2}}]
This is a proof for second lemma.

\end{proof}

Here should be a proper proof.

\end{proof}

\begin{rem}
Example of remark. Remarks, examples, notes and problems are displayed with non-italic font, like definitions, but with numbers.
\end{rem}

\begin{thebibliography}{99}

\bibitembook{gtwa}{J.A. Bondy, U.S.R. Murty}{Graph Theory with Applications}{North-Holland, NewYork-Amsterdam-Oxford}{1982}

\bibitemart{r3}{G. Chartrand, F. Harary and P. Zhang}{On the geodetic number of a graph}{Networks}{39}{2002}{1--6}

\bibitemproc{r5}{R.J. Gould, M.S. Jacobson and J. Lehel}{Potentially G-graphic degree sequences}{%
Combinatorics, Graph Theory, and Algorithms Vol. I}{Alavi, Lick and Schwenk}{New York: Wiley \& Sons, Inc.}{%
1999}{387--400}

\bibiteminpress{SK-11}{P. Sittitrai and  K. Nakprasit}{An analogue of DP-coloring for variable degeneracy and its applications}{Discuss. Math. Graph Theory}{2019}\\
\doi{10.7151/dmgt.2238}

\bibitemarxiv{cela2019monotonic1}{E. Cela and E. Gaar}{Monotonic representations of outerplanar graphs  as edge intersection graphs of paths on  a grid}{2019}{1908.01981}

\bibitemurl{oeis11}{N.J.A. Sloane}{The {O}n-{L}ine {E}ncyclopedia of {I}nteger {S}equences}{2021}{https://oeis.org}

%% manuscript
\bibitemman{hopI1}{S.K. Ayyaswamy and C.~Natarajan}{Hop domination in graphs}{2015}

%% other bibliography item...
\bibitemrem{bibr}{L. Ponomarenco}{Stochastic integral with respect to the multiparameter Brownian motion and attached stochastic equations}{1972}{in Russian}

\end{thebibliography}


\end{document}