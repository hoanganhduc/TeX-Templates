\documentclass{article}

\newcommand \journalacronym{dmgt}
\newcommand \JOURNALACRONYM{DMGT}
\newcommand \volumetitle{Graph Theory}
\newcommand \fullvolumetitle{Discussiones Mathematicae Graph Theory}
\newcommand \voldoinbr{dmgt}
\def\mscyear{2020}


%\newcommand \clsname{\JOURNALACRONYM{}}
\newcommand \clsname{\journalacronym{}}
\newcommand \clsnamex{\ttcomm{\clsname{}}}
\newcommand \journal{\textit{\fullvolumetitle}}


\usepackage{color}

\usepackage{section}
\def\hddot{.} % after inline headings, theorems etc
\def\cpdot{:} % in figure/table captions - also apears in list of figures
\def\nmdot{.} % after section numbers etc

\newcommand{\bs}{{$\backslash$}}
\newcommand{\ttcomm}[1]{{\ttfamily{#1}}}
\newcommand{\comm}[1]{{\textcolor{blue}{{\bs}#1}}}
\newcommand{\cmpar}[1]{{$\{$}{#1}{$\}$}}
\newcommand{\cmpary}[1]{{[{#1}]}}
\newcommand{\cmparz}[1]{{<{#1}>}}
\newcommand{\redbox}[1]{{\ttfamily{\textcolor{red}{#1}}}}
\newcommand{\vbl}[1]{\comm{#1}}
\newcommand{\vblx}[2]{{\redbox{\comm{#1}\cmpar{#2}}}}
\newcommand{\vblyx}[3]{{\redbox{\comm{#1}\cmpary{#2}\cmpar{#3}}}}
\newcommand{\vblxy}[3]{{\redbox{\comm{#1}\cmpar{#2}\cmpary{#3}}}}
\newcommand{\vblxx}[3]{{\redbox{\comm{#1}\cmpar{#2}\cmpar{#3}}}}
\newcommand{\vblxxxy}[5]{{\redbox{\comm{#1}\cmpar{#2}\cmpar{#3}\cmpar{#4}\cmpary{#5}}}}
\newcommand{\vblxyx}[4]{{\redbox{\comm{#1}\cmpar{#2}\cmpary{#3}\cmpar{#4}}}}
\newcommand{\vblz}[1]{{\redbox{\cmparz{#1}}}}
\newcommand{\vblzx}[3]{{\redbox{\comm{#1}\cmparz{#2}\cmpar{#3}}}}
\newcommand{\vblzxx}[4]{{\redbox{\comm{#1}\cmparz{#2}\cmpar{#3}\cmpar{#4}}}}
\newcommand{\vblzxxx}[5]{{\redbox{\comm{#1}\cmparz{#2}\cmpar{#3}\cmpar{#4}\cmpar{#5}}}}
\newcommand{\vblxxxxxxxx}[9]{{\redbox{\comm{#1}\cmpar{#2}\cmpar{#3}\cmpar{#4}\cmpar{#5}\cmpar{#6}\cmpar{#7}\cmpar{#8}\cmpar{#9}}}}
\newcommand{\vblxxxxxxx}[8]{{\redbox{\comm{#1}\cmpar{#2}\cmpar{#3}\cmpar{#4}\cmpar{#5}\cmpar{#6}\cmpar{#7}\cmpar{#8}}}}
\newcommand{\vblxxxxxx}[7]{{\redbox{\comm{#1}\cmpar{#2}\cmpar{#3}\cmpar{#4}\cmpar{#5}\cmpar{#6}\cmpar{#7}}}}
\newcommand{\vblxxxxx}[6]{{\redbox{\comm{#1}\cmpar{#2}\cmpar{#3}\cmpar{#4}\cmpar{#5}\cmpar{#6}}}}
\newcommand{\vblxxxx}[5]{{\redbox{\comm{#1}\cmpar{#2}\cmpar{#3}\cmpar{#4}\cmpar{#5}}}}
\newcommand{\vblxxx}[4]{{\redbox{\comm{#1}\cmpar{#2}\cmpar{#3}\cmpar{#4}}}}


\author{Grzegorz Arkit}
\title{How to use \clsname{} style}

\begin{document}
\maketitle

\begin{abstract}

\clsname{} style is intended for the preparation of articles which will be published in \journal{}. 
The style is designed in such a way that its usage is simple and the rules which should be followed are similar 
to the standard rules for publications in system \LaTeX{}.
\end{abstract}

\section{General rules}
\begin{itemize}
 

	\item in every command one should not use spaces between the brace brackets and parameters. In case when, there is a need to transfer the text, then after the brace bracket the comment sybol (\%) must be inserted and after that, one can start a new line (the example is in section The bibliography - command bibitemproc).\\
	\item we recommend the full versions of  MiKTeX or TeXLive, because \clsname{} style requires the following packages: section,  amsthm,  textcase, setspace, amssymb,  lineno.\\
	\item Since the numeration of pages changes during the editorial process, one cannot use the references to pages eg. "theorem on page 3".\\
\end{itemize}
\section{Setting the name of the class}\label{sec1}

Every article which uses our style should begin with the call of the class \clsname{}:\\
\vblx{documentclass}{\journalacronym{}} \\
or\\
\vblyx{documentclass}{options}{\journalacronym{}}.


As \ttcomm{options} one can set the parameters:
\begin{itemize}
\item \ttcomm{thmsec} -- the theorems will be enumerated within the section, if one omitt this option, then the theorems will be enumerated continously in the whole article (default option).  
\item \ttcomm{dntnbr} -- the definitions will be enumerated together with the theorems (this is not a default option).
\item \ttcomm{note} --  the word {\sc Note} will appear before the title.
\item \ttcomm{prblcol} -- the caption {\sc problems Column} will appear before the name of the magazine.
\end{itemize}


One can choose more than one from the above options (except both of the last ones) by separating them with a comma.

\section{Setting the data of the paper  - authors, title, keywords, MSC} 
 
To set the title of the paper one must use the command \vblx{title}{Article title}.
The title is automatically converted to upper case, that is why if we want to have e.g. footnotes in the title, then we must use a special command. Otherwise also footones will be converted to upper case:\\

\noindent
\redbox{\comm{def}\bs{}tfn\cmpar{\bs{}footnote\{Footnote\}}}\\
\vblxy{title}{Article title \vbl{tfn}}{Article title}.\\

On the example there is presented the usage of a version of the command \ttcomm{title} with an additional parameter, which will be used as a title in the heading of the page.
If we use the basic version of this command, then on every page which has a heading with a title, the footnote will also appear.


The version with an optional parameter is also useful in case when the whole title does not fit to page. Then one can choose a proper part of the title which will appear in the heading of the page.


After the title we add the authors of the paper. In principle every author should be added with a command:\\
\vblxxxy{newauthor}{full author name}{short author name}{affiliation}{email}.

After the title we add the authors of the paper. In principle every author should be added with a command:\\
\vblxxxy{newauthor}{full author name}{short author name}{affiliation}{email}.


In the fields
\ttcomm{full author name}, \ttcomm{short author name} and \ttcomm{email} we can use brake-line commands (\bs{}\bs{}).


The parameter \ttcomm{short author name} is used for automatic inserting the list of the authors in the heading of the page.


Similarly as title also footnotes must be defined with a command.


The list of the authors is build automatically. In case a larger number of the authors, before the last one the word {\sc and} is added on the title page and in the heading of the page.

Sometimes there is a need to make authors list in a different way, then we can use the brake-line commands.

If that is not enough (e.g. the list of the authors is too long), then we can redefine the command \vbl{authornames} to modify the list of the authors which will appear in the heading:\\
\vblxx{def}{\bs{}authornames}{Author 1, Author 2 and Author 3}.

In the preamble one must define keywords and MSC. For this purpose we have two commands
\vblx{keywords}{value} and \vblx{classnbr}{value}.


All data defined in the preamble are automatically inserted into the document:
\begin{itemize}
\item title and authors at the beginning of the document
\item keywords and MSC at the end of the abstract.
\end{itemize}



\section{Theorems, proofs, definitions}


The following theorem environments are defined:
\begin{itemize}
\item \ttcomm{theorem} -- Theorem 
\item \ttcomm{lemma} -- Lemma
\item \ttcomm{prop} -- Proposition
\item \ttcomm{obs} -- Observation
\item \ttcomm{cor} -- Corollary
\item \ttcomm{con} -- Conjecture
\end{itemize}

The next environments are handled as theorems but are displayed in a different style (without italic font - like definitions):
\begin{itemize}
\item \ttcomm{rem} -- Remark
\item \ttcomm{note} -- Note
\item \ttcomm{exm} -- Example
\item \ttcomm{prb} -- Problem
\end{itemize}

\noindent For the definitions there is also a special environment  \ttcomm{dnt}. If there is such need then one can define additional  environments\\
\noindent\textbf{theorem-like}\\
\vblx{theoremstyle}{plain}\\
\vblxyx{newtheorem}{name}{theorem}{Title}\\
\noindent \textbf{definition-like}\\
\vblx{theoremstyle}{definition}\\
\vblxyx{newtheorem}{name}{theorem}{Title}.\\
Then the enumeration method will stay the same.
If one would not like to enumerate the new environment, then one must use the following version with a star of the command  \ttcomm{\bs{}newthorem*}. 
Then there is no need to give the optional parameter \ttcomm{\cmpary{theorem}}.

The way of enumerating theorems depends on the parameter \ttcomm{thmsec} which is specified in the document class (\ref{sec1}).

In the environments of theorems one can use optional parameter. However usually the optional parameter is put in round brackets, in our class these brackets are not showed - one must add them oneself.
This allows using the optional parameter in case when the additional information should appear without brackets - e.g. reference to the bibliography.

Examples: 
\begin{itemize}
\item typical use - \vblx{begin}{theorem} produces \\
{\bf Theorem 1.} ...
\item theorem or definition with name - \vblxy{begin}{theorem}{(Kuratowski)}
produces \\
{\bf Theorem 1 {\rm (Kuratowski)}.} ...
\item theorem or definition with reference -
\vblxy{begin}{theorem}{\bs{}cite\{label1\}} produces \\
{\bf Theorem 1 {\rm [1]}.} ...
\end{itemize}

Of course instead \ttcomm{theorem} one can use the name of any environment.

\subsection{Proofs}

For proofs there is an environment \ttcomm{proof}. It also can have an optional parameter.
This parameter will appear instead of the title of the proof (the full title - with the word \textit{\textbf{Proof}}- must be given).
If in the title there should be also a number, then one must use the following commands 
\vblx{begin}{proof}[Proof of theorem $\left\{\right.$\vblx{bf} 1]$\left.\right\}$.\\
The black square is used for denoting the end of the proof. Nested subproofs are ended with the empty square.


\section{The bibliography}

The bibliography should be defined with the environment \ttcomm{thebibliography} with the parameter (obligatory) \ttcomm{99}.
There are few commands which simplify the process of adding the most popular bibliography items.
In other cases one can use the standard command \ttcomm{bibitem}.

\begin{itemize}
\item article -- \vblxxxxxxx{bibitemart}{refname}{authors}{title}{magazine}{\%\\volume}{year}{pages},
\item book -- \vblxxxxx{bibitembook}{refname}{authors}{title}{publisher}{year},
\item article in proceedings -- \vblxxxxxxxx{bibitemproc}{refname}{authors}{title}{\%\\procname}{editors}{publisher}{year}{pages},
\item article in press-- \vblxxxxx{bibiteminpress}{refname}{authors}{title}{\%\\ magazine}{year},
\item manuscript -- \vblxxxx{bibitemman}{refname}{authors}{title}{year},
\item article (position) published in arXiv -- \vblxxxxx{bibitemarxiv}{refname}{\%\\authors}{title}{year}{arXiv ID},
\item article (position) published somewhere on Internet-- \vblxxxxx{bibitemurl}{\%\\refname}{authors}{title}{year}{url/address},
\item article (position) with some remarks-- \vblxxxxx{bibitemrem}{refname}{authors}{\%\\title}{year}{remarks}.
\end{itemize}
In particular, a \vbl{bibitemrem} can be helpful if none of the other options is fit for use.

\end{document}