%\documentclass[12pt,twoside]{article}
\documentclass[12pt]{article}

\usepackage[a4paper]{geometry}
\usepackage{ajc-new-ISSN-only}
\usepackage{amsmath,amsfonts,latexsym}
\usepackage{draftcopy}

\Volume{XY(2)}
\Year{2014}
\firstpage{123}
\lastpage{126}
\Received{today}
\Revised{tomorrow}
%%% The 6 lines above are completed by us later; the line
%%% below we modify. 
\runninghead{F. Bloggs et al.\,/\,Australas. J. Combin. XY\,(2) (2014), 123--126}

\parskip=.3em


\numberwithin{equation}{section}

\newtheorem{theorem}{Theorem}[section]
\newtheorem{corollary}[theorem]{Corollary}
\newtheorem{proposition}[theorem]{Proposition}
\newtheorem{definition}[theorem]{Definition}
\newtheorem{conjecture}[theorem]{Conjecture}
\newtheorem{lemma}[theorem]{Lemma}



\newenvironment*{proof}
{\begin{list}{}{\setlength{\leftmargin}{0em}\setlength{\rightmargin}{0em}}
\item[] {\sc Proof:}} {\hfill$\Box$
\end{list}}



\begin{document}

\title{This is the title of the paper and might be long,\\
in which case we decide where it should be split}
\author{Fred Bloggs\thanks{Supported by Antarctic Grant G12345}}
\Addr{Department of Oceanography\\
University of Here\\
Townplace\\
Country\\
{\tt fred.bloggs@ocean.gov}}

\authortwo{Amy Person \quad Joe Someone-same-address}
\Addrtwo{Department of Mathematics\\
University of Neverland\\
Thistown\\
Country\\
{\tt js@math.neverland \quad amyp@math.neverland}}

\authorthree{Josie Zimmer}
\Addrthree{Department of Oceanography\\
University of Here\\
Townplace\\
Country\\
{\tt josie.zimmer@ocean.gov}}

%%% ALL countries should be stated; we have noticed that a
%%% surprising number of papers from the U.S.A.\ fail to state their country!
%%% If an author requires two addresses, the second one should be placed
%%% in a ``thanks'' footnote.

\maketitle

\begin{abstract}Make sure that the abstract begins immediately after the closing right brace, 
with no space, so that the alignment of the abstract is correct. Please do not include
references such as [6] in the abstract; 
instead refer directly and briefly to papers, 
such as Jones [{\em J. Combin. Des.} 32 (1997), 23--45].
In this way, the abstract will stand alone.
Also please avoid long definitions in the abstract. 
An abstract should state briefly what the paper shows, 
and not have long details about what earlier papers have shown; 
those can be included in the introduction.
\end{abstract}

\section{Introduction}

This is the introduction to the paper.

Avoid capital letters in the paper title, 
apart from the initial letter and in proper names (such as ``Kirkman").
Mathematical text in the title and elsewhere must be written in \LaTeX\ math mode;
for instance, the title of \cite{hall+} is written 
\begin{center}\verb|On designs $(22,33,12,8,4)$|\,.\end{center}


\section{Some hints on typesetting}

Do not use ``iff"; write ``if and only if'' instead.
Note that the opening quote here is formed by two left single quotes.

\smallskip

Please do not use \ \verb|\\| \
for a newline or extra space; instead use
\verb|\smallskip| or 
\verb|\medskip| or even
\verb|\bigskip|.

You may set a \verb|\parskip|, although we reserve the right to change this if
the final page has very little on it.


If using ``e.g.'' or ``i.e.'' in a sentence, then please make sure that
you include the correct space afterwards; i.e.\ we add \verb|\| 
to ensure that the space is the same as after a word in a sentence
rather than the longer space at the end of a sentence.
The abbreviation {\sl et al.}\ is short for et aliter, 
and as in the case of ``e.g.'' you want
a backslash after the dot, in ``et al.'',
unless this occurs at the end of a sentence.

Use two dashes (\verb|--|)  between numbers,
especially between page numbers in the references.
Also, use three dashes (\verb|---|)  for a dash in a sentence --- like this one!
A~single dash (\verb|-|) is a hyphen as in the word X-ray or intra-word.

\begin{theorem} 
This is a theorem, and thanks to Kirkman,
we know that a Steiner triple system exists if and only if $v\equiv 1$ or $3 \ ({\rm mod}\ 6)$.
\end{theorem}

\begin{proof}
Here is the proof of the theorem.
We prefer an open square to indicate the end of a proof rather than a black square, 
especially if the black square is large!
\end{proof}

The final pagination and page breaks are determined by the machine on which 
we make our final version.
On rare occasions, different page breaks occur when a paper is run on different machines
which may be running slightly different versions of \LaTeX!

\section{Another section as required}

If the bibliography has ten or more references, 
then please  replace 
\begin{center}
\verb|\begin{thebibliography}{9}| \
\end{center}
by \ \verb|\begin{thebibliography}{99}|.

Ensure that references are in alphabetical order of the first author's last name.
Write ``A.B. Bloggs" rather than ``Bloggs, A.B.".
When there are two or more authors, 
precede the last author by ``and", with no comma before the ``and".

If you prefer a space between double initials, then please be consistent; 
we recommend a smaller space than that between words, such as
A.\,B. Bloggs (which is typed as \verb|A.\,B. Bloggs|).

{\it Always\/} use two dashes between page numbers: so 12--14 and {\it not\/} 12-14.

Only include references which are actually cited within the paper. (This skeleton file fails
that requirement!)

When writing dots, as in a list, type \verb|\dots| rather than three dots;
note the difference between $1,2,3,\dots$ and $1,2,3,...$ for instance.
Also avoid using \verb|\cdots|; it is usually
best to let \LaTeX\ decide itself where to place the dots!

Here is a list of items:
\begin{enumerate}
\item
The first item.
\item
The second item.
\item
If a mathematical term is abbreviated and involves two or three letters,
as in ``IRS'' for the irredundance saturation number of a graph $G$,
then it should {\em not\/} be typeset in math mode, but should be in roman font, as in
\[
  {\rm IRS}(G) = \min\{{\rm IRS}(v,G) \mid v \in V\}\,.
\]
\end{enumerate}
Here is some displayed mathematics:
\[
  27x^2-15y+p^3 = 234x-\sqrt{yp}\,.
\]
For several lines of equations, you can use ``eqnarray'', as follows:
\begin{eqnarray*}
  x &=& 27z^2\\
  y &>& a+b+c+d+e\,.
\end{eqnarray*}
The asterisk here in the .tex file means that the equations are not numbered.


\section{Further reading}

The pages
\begin{center}
\tt http://web.maths.unsw.edu.au/$\sim$michaelc/mc\_writing.pdf
\end{center}
and
\begin{center}
\tt http://web.maths.unsw.edu.au/$\sim$michaelc/lms\_writing.pdf
\end{center}
include excellent comments on writing mathematics well.


\begin{thebibliography}{9}

\bibitem{FredB}
F. Bloggs and J. Zimmer,
This is the end of the world,
{\em J. Oceanography} 13 (2000), 13--21.

\bibitem{bb+erdos}
B. Bollob\'as and P. Erd\H os,
Graphs of extremal weights,
{\em Ars Combin.} 50 (1998), 225--233.

\bibitem{hall+}
M. Hall Jr., R. Roth, G.\,H.\,J. van Rees and S.\,A. Vanstone,
On designs $(22,33,12,8,4)$,
{\em J. Combin. Theory Ser. A} 47 no.\,2 (1988), 157--175.

\bibitem{book}
F. Kloggs, {\it This is a book title in italics},
Cambridge University Press, 2010.

\end{thebibliography}

\end{document}



