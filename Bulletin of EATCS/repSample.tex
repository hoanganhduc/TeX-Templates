\documentclass[final]{beatcs}
\begin{document}
\beatcsReport{PSI 2003}
 {The 5th A.P. Ershov International\\Conference on
   Perspectives of System Informatics}
 {Alexandre Zamulin\\A. P. Ershov Institute of Informatics Systems}

 PSI 2003, the 5th in the series of international conferences
 ``PERSPECTIVES OF SYSTEM INFORMATICS'' aimed to honor the 70th
 anniversary of A.P.~Ershov's closest colleague, the late professor
 Igor Pottosin (1933--2001), and his outstanding contributions towards
 advancing informatics took place in Novosibirsk Akademgorodok from
 July 9 to July 12, 2003. The conference was organized by the
 A.P.~Ershov Institute of System Informatics of the Siberian Division
 of the Russian Academy of Sciences. Alexander Marchuk, director of
 this institute, was the conference chair and Natalya Cheremnyh was
 the conference secretary. 

 Akademgorodok (Academic town) is a big research center, 30 km South
 from Novosibirsk, the largest city of Siberia. Akademgorodok is
 located in a picturesque place near the Ob lake. It is surrounded
 with birch and pine forests and pleasant not only for work but for
 recreation as well. Silence, beautiful landscape, and pure air are
 the factors promoting scientific activity and creativity. One can
 obtain more information about Akademgorodok on the web site
 \url{http://www.nsc.ru/other/akadem/index.htm}.

 Andrei Ershov was one of the early Russian pioneers in the field of
 theoretical and systems programming, a founder of the Siberian School
 of Computer Science. His closest colleague, Igor Pottosin, worked for
 the Siberian Branch of Russian Academy of Sciences since 1958, step
 by step filling positions from junior researcher to director of
 A.P.Ershov Institute of Informatics Systems. In recent years he
 headed the Laboratory of Systems Programming in this Institute and
 the Department of Programming at Novosibirsk State University. 

 I.V.~Pottosin took a leading position among Russian specialists in
 computer science. It is hardly possible to overestimate his
 contribution to the formation and development of this research
 direction in this country. He has obtained fundamental results in the
 theory of program optimization, formulated main principles and
 typical schemes of optimizing compilers, and suggested efficient
 algorithms of optimizing program transformations. One of the world's
 first optimizing compilers ALPHA, the ALPHA-6 programming system, and
 the multi-language compiling system BETA has been designed and
 developed on the basis of these results and with direct participation
 of Igor Pottosin. In recent years he mainly concentrated on the
 problems of designing programming environments for efficient and
 reliable program construction and headed the SOKRAT project aimed at
 producing software for embedded computers. Research in program
 analysis and programming methodology was essentially inspired by this
 practical work, which greatly influenced its results. Prof. Pottosin
 was actively involved in the training of computer
 professionals. There is a professor, eleven Ph.D.\ holders, and
 hundreds of graduates among his disciples.

 The first four Andrei Ershov conferences were held in Akademgorodok
 in May 1991,  June 1996, July 1999, and July 2001, respectively, and
 proved to be remarkable international events. The program of the
 fifth conference included many of the topics of the fourth one:
 theoretical computer science, programming methodology, and new
 information technologies, which are the most important constituents
 of system informatics. The style of the previous conferences was
 preserved to a certain extent: a considerable number of invited
 papers in addition to contributed papers.  

 The Programme Committee consisted of 50 members from 23
 countries. It was co-chaired by Manfred Broy (Germany) and Alexandre
 Zamulin (Russia).

 This time 110 papers (a record number!) were submitted to the
 conference by researchers from 30 countries. Each paper was reviewed
 by three experts, at least two of them from the same or closely
 related discipline as the authors. The reviewers generally provided
 high quality assessment of the papers and often gave extensive
 comments to the authors for the possible improvement of the
 presentation. As a result, the Program Committee selected 25 high
 quality papers as regular talks and 28 papers as short talks.  

 A broad range of hot topics in system informatics was covered by the
 following invited talks: 
\begin{enumerate}\itemsep0pt plus 4pt minus 2pt
 \item \textbf{\textit{Kim Bruce}} ((Williams College, USA). 
        ``Bending without breaking: Making software more flexible.''   
 \item\textbf{\textit{David Harel}} (The Weizmann Institute of Science,
      Israel). ``On the Visualization and Aesthetics of Large
      Graphs.'' 
\item\textbf{\textit{Tony Hoare}} (Microsoft Research, UK). 
        ``The Verifying Compiler: a Grand Challenge for Computing
        Research.''
 \item\textbf{\textit{Bertrand Meyer}}  (ETH Zurich, Switzerland, and
    Eiffel Software, USA). ``How I Teach Programming.''
\end{enumerate}

 The whole number of participants was 125 with the following
 distribution by countries: 
        Australia: 2, Austria: 1, Belgium: 3, Denmark: 2, Estonia: 1,
        France: 3, Germany: 10, Israel: 1, Italy: 3, Japan: 2, 
        Mexico: 4, the Netherlands: 4, Norway: 1, Spain: 4, 
        Russia: 73, Sweden: 1, Switzerland: 1, UK: 6, USA: 3. 

 Conference program consisted of seventeen sessions: 
 {\scshape 
   Software Education, Software Engineering, Programming Issues,
   Graphical Interfaces, Partial Evaluation \&{} Supercompilation,
   Verification, Program Synthesis, Transformation \&{} Semantics,
   Logic and Types, Concurrent \&{} Distributed Systems, Concurrent \&{}
   Reactive Systems, Program Specification, Verification \&{} Model
   Checking, Documentation \&{} Testing, Databases, Constraint
   Programming, Natural Language Processing}. 
 Details of the program can be found on the web site
 \url{http://www.iis.nsk.su/PSI03}.

 The participants stayed at the hotel ``Zolotaya Dolina'' (Golden
 Valley) situated in the center of Akademgorodok. All conference
 sessions were held in the Small Hall of the House of Scientists
 located  up the street from the hotel --- approximately a five minute
 walk.  

 Breakfast was served in the hotel restaurant. Lunch was offered in
 the restaurant of the House of Scientists. The guests enjoyed
 traditional Russian cuisine, including famous borshch. During coffee
 breaks various choices of sandwiches and Russian pirozhki were
 available. 

 An informal picnic for the ``early birds'' at a beach of the Ob lake
 was held the night preceding the first day of the conference. The
 participants enjoyed wine, snacks, and night swimming in the lake.  

 The \textit{welcome party} took place the first day of the conference
 in the restaurant of the House of Scientists. It lasted till midnight. A
 music band and a good collection of discs provided a wonderful
 environment for dances. 

 The next day a \textit{barbecue picnic} with Russian steam bath took place
 at the Blue Lake villa, 10 km away from Akademgorodok. It lasted till 10
 p.m. People tasted a good Moldavian wine and Russian shashlyk
 (shish-kebab) and enjoyed taking the Russian steam bath and swimming
 in the Ob Lake.  

 The \textit{conference dinner} took place the last day of the conference in
 the restaurant of the Zolotaya Dolina hotel. In the first part of the
 dinner there were plenty of speeches following ``go to'' procedure
 (each speaker designated the next speaker). In the second part of the
 dinner people mainly danced to the accompaniment of a music band. The
 dinner finished at 1 a.m.  

 The participants also enjoyed a number of excursions to Novosibirsk
 downtown, Geological Museum (one had a chance to see there a
 wonderful collection of Siberian minerals and treasure stones),
 Museum of the Cultural History of Siberian and Far East Peoples (one
 had a chance to see ``mumia'' there), and Railway Steamer Museum. 

 The conference was followed by International Workshop on Program
 Understanding held in a picturesque place of Altai mountains. The
 workshop page is 
 \url{http://www.iis.nsk.su/psi03/workshop/index_e.shtml}.
 

 PSI 2003 was a very well organized high-level conference. The last
 day the participants thanked the organizers for the good organization
 of the conference and expressed their desire to come to Akademgorodok
 once again. 

\end{document}