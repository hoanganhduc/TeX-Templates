\documentclass{beamer}

\usepackage{hus-beamer} % for inserting HUS logos at the top-right and bottom-left corners, modify hus-beamer.sty for removing or re-positioning the logos
\hypersetup{
    unicode=true,          % non-Latin characters in Acrobat’s bookmarks
    pdfencoding=unicode,
    pdftoolbar=true,        % show Acrobat’s toolbar?
    pdfmenubar=true,        % show Acrobat’s menu?
    pdffitwindow=false,     % window fit to page when opened
    pdfstartview={FitH},    % fits the width of the page to the window
    pdftitle={HUS-Beamer Template},    % title
    pdfauthor={Hoàng Anh Đức},     % author
    pdfsubject={HUS-Beamer Template},   % subject of the document
    pdfcreator={Hoàng Anh Đức},   % creator of the document
    pdfproducer={XeLaTeX}, % producer of the document
    pdfnewwindow=true,      % links in new window
    colorlinks=false,       % false: boxed links; true: colored links
    linkcolor=false,          % color of internal links (change box color with linkbordercolor)
    urlcolor=false}
    
\usepackage[vietnamese]{babel} % Vietnamese in LaTeX

\mode<presentation>{
	\usefonttheme{professionalfonts} % normal font for math formulas
	% insert section page with title only
	% before each section
	\AtBeginSection[]{
	\begin{frame}%[noframenumbering] % remove this if you do not want to number section page
	\vfill
	\centering
	\begin{beamercolorbox}[sep=8pt,center,shadow=true,rounded=true]{title}
	\usebeamerfont{title}\insertsectionhead\par%
	\end{beamercolorbox}
	\vfill
	\end{frame}
}
}

% Some common packages
\usepackage{amsmath}
\usepackage{amsfonts}
\usepackage{amssymb}

\setbeamertemplate{theorems}[numbered] % numbering theorem environment

% For Vietnamese

\deftranslation[to=vietnamese]{Theorem}{Định lý}
\deftranslation[to=vietnamese]{Corollary}{Hệ quả}
\deftranslation[to=vietnamese]{Problem}{Bài tập}
\deftranslation[to=vietnamese]{Solution}{Lời giải}
\deftranslation[to=vietnamese]{Definition}{Định nghĩa}
\deftranslation[to=vietnamese]{Lemma}{Bổ đề}
\deftranslation[to=vietnamese]{Example}{Ví dụ}

%%% New environment

\theoremstyle{plain} % other style are definition and example, you can also define your own style
\newtheorem{proposition}{Mệnh đề}

%%% Numbering examples with a separate counter, comment the below if not use
\usepackage{etoolbox} % for \undef command => re-define environment
\undef{\problem}
\newtheorem{problem}{Bài tập}
\undef{\example}
\theoremstyle{example} % style of the example environment
\newtheorem{example}{Ví dụ}

% Biblatex in beamer
\usepackage[bibstyle=authoryear, citestyle=authoryear, maxcitenames=2, maxbibnames=100, backend=bibtex]{biblatex}

\AtBeginBibliography{\footnotesize} % Footnotesize for Bibliography entries

\setbeamertemplate{bibliography item}{%
  \ifboolexpr{ test {\ifentrytype{book}} or test {\ifentrytype{mvbook}}
    or test {\ifentrytype{collection}} or test {\ifentrytype{mvcollection}}
    or test {\ifentrytype{reference}} or test {\ifentrytype{mvreference}} }
    {\setbeamertemplate{bibliography item}[book]}
    {\ifentrytype{online}
       {\setbeamertemplate{bibliography item}[online]}
       {\setbeamertemplate{bibliography item}[article]}}%
  \usebeamertemplate{bibliography item}}

\defbibenvironment{bibliography}
  {\list{}
     {\settowidth{\labelwidth}{\usebeamertemplate{bibliography item}}%
      \setlength{\leftmargin}{\labelwidth}%
      \setlength{\labelsep}{\biblabelsep}%
      \addtolength{\leftmargin}{\labelsep}%
      \setlength{\itemsep}{\bibitemsep}%
      \setlength{\parsep}{\bibparsep}}}
  {\endlist}
  {\item}

\addbibresource{refs.bib}

\defbibheading{bibliography}[\refname]{}

% Info
\title{HUS-Beamer Template}
\author{Hoàng Anh Đức}
\institute[Short Institute]{\texttt{email@address.com}\\ Long Institute}
\date{18/10/2018}

\begin{document}

\begin{frame}[plain,noframenumbering]
\placelogofalse % No logo at the title page
\titlepage
\end{frame}

\placelogotrue % Logo at other pages

\section{Test some common Beamer elements}

\frame{
	\frametitle{Math}
	
	Inline Math $E = mc^2$.
	
	Numbered Equation
	\begin{equation}
	E = mc^2
	\end{equation}
	
	Listing
	\begin{itemize}
	\item One
	\item Two
	\item Three
	\end{itemize}
	\begin{enumerate}
	\item One
	\item Two
	\item Three
	\end{enumerate}
}

\begin{frame}[fragile]
	\frametitle{Theorem, Lemma, etc.}
	
	\begin{theorem}[Handshaking Theorem]
	\[
		\sum_{v \in V(G)}\deg_G(v) = 2\vert E(G) \vert
	\]
	\end{theorem}
	
	\begin{problem}
	A problem.
	\end{problem}
	
	\begin{example}
	An example
	\end{example}
	
	Other environments supported by Beamer are: \verb+corollary+, \verb+fact+, \verb+lemma+, \verb+solution+, \verb+definition+, \verb+definitions+, \verb+examples+.
\end{frame}

\begin{frame}[fragile]
	\frametitle{Blocks}
	\begin{block}{This is an example of a block}
	When using \verb+verbatim+ in Beamer for inserting codes, it is better to put them between \verb+\begin{frame}[fragile]+ and \verb+\end{frame}+, other than \verb+\frame[fragile]{+ and \verb+}+.
	\end{block}
\end{frame}

\begin{frame}[fragile]
	\frametitle{Citation}
	
	See~[\cite{Rosen2011}] (using \verb+\cite+ command), which is also
	
	\fullcite{Rosen2011}
	
	(using \verb+\fullcite+ command).
	
\end{frame}

\section{Second Section}

\section{Bibliography}
\frame[allowframebreaks]{
	\frametitle{Bibliography}
	% For Natbib
%	  \def\newblock{\hskip .11em plus .33em minus .07em} % important line
%	\bibliographystyle{plainnat}
%    \bibliography{Bibliography}

	% For Biblatex
	\printbibliography
}
\end{document}
